\documentclass[a4paper,11pt]{article}
\usepackage{amsmath,amsfonts,amssymb,amsthm,epsfig,epstopdf,titling,url,array}
\usepackage[utf8]{inputenc} 
 \usepackage{thmtools}

\theoremstyle{plain}
\newtheorem{thm}{Thm}[section]
\newtheorem{lem}[thm]{Lemma}
\newtheorem{oss}[thm]{Osservazione}
\newtheorem*{cor}{Corollario}

\theoremstyle{definition}
\newtheorem{defn}{Def}[section]
\newtheorem{conj}{Conjecture}[section]
\newtheorem{exmp}{Example}[section]

\theoremstyle{remark}
\newtheorem*{rem}{Remark}
\newtheorem*{note}{Note}
\usepackage{hyperref}

\begin{document}
\title{Appunti sintetici di calcolo delle probabilità \\(corso progredito)}
\maketitle
\footnote{Appunti interamente tratti dal libro “Elementi di statistica delle decisioni” del prof. emerito Silvano Holzer disponibile su \url{https://www.openstarts.units.it/bitstream/10077/2743/1/testoHolzer.pdf}{} }

\footnote{L'indice analitico potrebbe risultare eccessivamente specifico, ma ciò è stato fatto allo scopo di avere sottomano l'ossatura  del corso, così da agevolare il ripasso finale. Alcuni nomi di teoremi sono stati inventati}


%%%%%%%%%%%
% Personalizzazione dei lucidi del prof a cura di Luca Buratto 
%%%%%%%%%%

\newpage

\tableofcontents
%\chapter{}
%\chapter{}
%\chapter{}
%\chapter{Convergenza di variabili aleatorie}
\newpage
%==================================
%DISPENSA 1
%===================================
\section{Integrale di Lebesgue}
Fissiamo un insieme ambiente e chiamiamo $\boldsymbol{\sigma}$\textbf{-algebra} (su $\Omega$) ogni famiglia $\mathcal{A}$ di sottoinsiemi di $\Omega$ tale che\\
$-\Omega\in \mathcal{A}$\\
$-A\in \mathcal{A}\Rightarrow A^{c}\in \mathcal{A}$\\
$-A_{n}\in \mathcal{A}$ per tutti gli $n\displaystyle \geq 1\Rightarrow\bigcup_{n\geq 1}A_{n}\in \mathcal{A}$

cio\`{e} tale che contenga l'insieme ambiente e sia chiusa per complementazione e unioni discrete. Ricorrendo alle formule di De Morgan, otteniamo la definizione equivalente (avendo denotato, come sempre faremo, con $A_{i}$ elementi generici di $\mathcal{A}$):

$-A_{1}\backslash A_{2}\in \mathcal{A}$;

$-\emptyset, \Omega\in \mathcal{A}$;

$-\displaystyle \bigcup_{i\in I}A_{i}, \displaystyle \bigcap_{i\in I}A_{i}\in \mathcal{A}$, se $I$ \`{e} un insieme discreto.

\subsection{Sigma-algebra generata}

\begin{defn}
$\sigma$\textbf{-algebra generata} dalla famiglia $\mathcal{G}\subseteq 2^{\Omega}$.È la $\sigma$-algebra su $\Omega$:

$\sigma(\mathcal{G})=\cap$\{ $\mathcal{A}$: $\mathcal{G}\subset \mathcal{A}$ e $\mathcal{A}$  $\sigma$-algebra su $\Omega$\}
\end{defn}
In parole povere  è la più piccola sigma algebra che contiene $\mathcal{G}$ \\
Ovviamente, se $\mathcal{G}\subseteq \mathcal{G}',\sigma(\mathcal{G})\subseteq\sigma(\mathcal{G}')$  e $\sigma(\mathcal{G})=\mathcal{G}$, se $\mathcal{G}$ è una $\sigma$-algebra.

\subsection{Traccia della sigma-algebra su S}
\begin{defn}
\textbf{Traccia della $\sigma$-algebra $\mathcal{A}$ su $ S\subset\Omega$} è la $\sigma$-algebra su $S$ formata da da tutti gli insiemi che stanno in sia in A che in S:
$$
\mathcal{A}\cap S=\{A\cap S:A\in \mathcal{A}\}
$$
\end{defn}

\subsection{Sigma-algebra indotta da una applicazione}
\begin{defn}
\textbf{$\sigma$-algebra indotta da una applicazione $\tau$} : $\Omega_{0}\neq\emptyset \rightarrow \Omega$ \text{È} la $\sigma$-algebra su $\Omega_{0}$:
$$
\tau^{-1}(\mathcal{A})=\{\tau^{-1}(A):A \in \mathcal{A}\}
$$
formata dalle controimmagini degli elementi della $\sigma$-algebra $\mathcal{A}.$
\end{defn}

\subsection{Retta reale ampliata}
\begin{defn}
$\mathbb{R}^{*}=\mathbb{R}\cup\{-\infty,\ +\infty\}$ ottenuta aggiungendo all'insieme $\mathbb{R}$ dei numeri reali
(retta reale) $\mathrm{i}$ simboli $-\infty$ e  $+\infty.$
\end{defn}

$\bullet$ Data questa aggiunta, dobbiamo escludere le  espressioni aritmetiche che coinvolgono rapporti di infiniti o rapporti con denominatori nulli, oppure
somme di infiniti di segno opposto o differenze di infiniti di ugual segno, poiché sono forme indeterminate.


\subsection{sigma-algebra di Borel}
\begin{itemize}
    \item[$\bullet$]  La $\sigma$-algebra di Borel (di $\mathbb{R}$) è la $\sigma$-algebra $\mathcal{B}$ su $\mathbb{R}$ generata dalla famiglia degli intervalli limitati inferiormente aperti $\mathrm{e}$ superiormente chiusi, cioè
$$\mathcal{B}=\sigma(\{]a,\ b]\ :\ a<b,\ a,\ b\in \mathbb{R}\})$$
La sua introduzione, fatta nel 1898 da Emile Borel, inaugurò una nuova era dell’analisi
matematica fornendo il punto di partenza sia di una classificazione topologica degli insiemi
di punti che della formulazione astratta della nozione d’integrale.\\

$\mathcal{B}$ è anche la $\sigma$-algebra generata da una qualsiasi famiglia d'intervalli (limitati o no) del medesimo tipo.\\

$\mathcal{B}$ è anche generata dalla famiglia $\mathcal{U}$ degli insiemi aperti di $\mathbb{R}$ e quindi anche dalla famiglia $\{W^{c}:W\in \mathcal{U}\}$ degli insiemi chiusi.\\

\item[$\bullet$] La $\sigma$-algebra di Borel (di $\mathbb{R}^{*}$) è la $\sigma$-algebra $B^{*}$ su $\mathbb{R}^{*}$ generata dalla famiglia degli intervalli di $\mathbb{R}^{*}$ inferiormente aperti $\mathrm{e}$ superiormente chiusi.\\

Pertanto, $B^{*}$ è pure generata dalla famiglia degli insiemi aperti della retta reale ampliata.\\

$B^{*}$ è anche generata dalla famiglia $\{[-\infty,\ a]\ :\ a\in \mathbb{R}\}$ delle semirette inferiori della retta reale ampliata di origine un numero reale.
\end{itemize}
Infine sappiamo che vale $B\subset B^{*}$\\

Gli elementi di {\it B} e di $B^{*}$ vengono chiamati rispettivamente \textbf{boreliani} della retta reale $\mathrm{e}$ della retta reale ampliata.

Richiamo brevemente la nozione di probabilità così poi da estenderla e vederne le analogie.

\begin{defn}
 Una probabilità su una $\sigma$-algebra di eventi è una  applicazione  $\mathrm{P}\mathrm{r}:\mathcal{F}\rightarrow [0,1]$ tale che:
\begin{itemize}

\item[•]$\mathrm{P}\mathrm{r}(\Omega)=1$ $\rightarrow$ $\mathrm{P}\mathrm{r}(\emptyset)=0$

\item[•]$\displaystyle \mathrm{P}\mathrm{r}(\bigvee_{n \geq 1}{E_{n})}=
\sum_{n\geq 1}\mathrm{P}\mathrm{r}(E_{n})$
 per ogni successione $(E_{n})_{n\geq 1}$ di eventi a due a due incompatibili.
\end{itemize}
\end{defn}
(ocio: probabilità definita su sigma algebra! Non su omega)

\subsection{Misura}
\begin{defn}
Con riferimento ad una $\sigma$-algebra arbitraria $\mathcal{A}$, una \textbf{misura} su $\mathcal{A}$ è un'applicazione $\mathrm{m}:\mathcal{A}\rightarrow[0,\ +\infty]$  se:
\begin{itemize}

\item[•]$\mathrm{m}(\emptyset)=0$ la misura dell'insieme vuoto vale zero

\item[•]m $(\displaystyle \bigcup_{n\geq 1}A_{n})=\sum_{n\geq 1}\mathrm{m}(A_{n})$ per ogni successione disgiunta $(A_{n})_{n\geq 1}$\\ \textbf{numerabile additività}
\end{itemize}
\end{defn}



\paragraph{Introduzione alla misura di Lebesgue} Preso un intervallo di estremi $a,b  $ con $a \leq b$ la lunghezza dell'intervallo è $b-a$. Se prendo una sequenza finita di intervalli $I_n$ a due a due disgiunti $a_1,b_1$; $a_2,b_2$; $...$; $a_n,b_n$ quanto misura la lunghezza della loro unione? La lunghezza della loro unione è pari alla somma delle lunghezze lg$(\bigvee_{n \geq 1}{I_n})=\sum_{i=1}^{n}{(b_i-a_i)}$. Se prendo una semiretta di estremi $a, +\infty$ (oppure $-\infty, a$) quanto sarà lunga la semiretta? Sarà lunga $+\infty$, ecco che interviene la condizione vista nell'altra pagina ossia che la misura può valere anche $+\infty$.\\

\textbf{Teor. Lebesgue} Si può dimostrare -risultato della tesi di dottorato di Lebesgue- che partendo da certi insiemi elementari esiste una ed una sola misura $\lambda(\cdot)$ sui boreliani tale che questa misura calcolata su un qualsiasi intervallo è proprio la lunghezza dell'intervallo:
$$
\exists! \,\, \lambda(\cdot) \mathrm{\,\, su \,\,} \mathcal{B} \mathrm{\,\, t.c. \,\,} \lambda(\mathrm{intervallo})=\mathrm{lunghezza (intervallo)}
$$
Questa misura è la generalizzazione del concetto di lunghezza.\\
Ma siamo sicuri che l'insieme dei boreliani coincida con l'insieme delle parti? Cioè che la $\mathcal{B}$ raggiunga tutti i sottoinsiemi?
Se fosse vero potremmo misurare la lunghezza di ogni insieme (intervalli, rette, semirette, ...). La risposta è no.
L'italiano Vitali ha dimostrato che esistono insiemi non boreliani per i quali non si può parlare di lunghezza, che sono cioè non misurabili; sono insiemi patologici. Perciò $\mathcal{B} \subsetneqq 2^\mathbb{R}$. Si può parlare di lunghezza solo sui borealiani, non su qualsiasi insieme di reali. Ma poco male perché gli insiemi di interesse per le applicazioni sono tutti misurabili.

\subsection{Misura di Lebesgue (unidimensionale)}

Considerata la famiglia $\Im=\{\emptyset, \mathbb{R},$ ]$a, b$], ]$-\infty, b], ]a, +\infty[$\}

-un insieme elementare $E$ è una unione finita di elementi di $\Im$ a due a due disgiunti

-per lunghezza dell'insieme elementare $E=$] $\alpha_{1}, \beta_{1}$] $\cup\cdots\cup$] $\alpha_{n}, \beta_{n}$] si intende l'elemento della retta reale ampliata:
$$
\mathrm{l}\mathrm{g}(E)=\sum_{i=1}^{n}(\beta_{i}-\alpha_{i})\ .
$$
Una misura di Lebesgue è una misura sui boreliani t.c. su unioni finite disgiunte di intervalli, semirette,.. essa è la usuale lunghezza. Quindi, solamente l'insieme vuoto ha lunghezza nulla. Mentre gli insiemi elementari illimitati sono gli unici di lunghezza infinita.\\

Considerata allora la famiglia:
$\mathcal{M} = \{S \subset \mathbb{R}| \lambda_{*}(S)=\lambda^{*}(S)\}$ degli \textbf{insiemi misurabili secondo Lebesgue}, la funzione $\lambda: \mathcal{M} \rightarrow [0,+\infty]$ così definita 
$$ 
\lambda(S)=\lambda_{*}(S)=\lambda^{*}(S)\}
$$ è una estensione ad insiemi piu complessi della nozione elementare di lunghezza, essendo $\lambda(E)$=lg$(E)$.  \\
(Date le misure interna ed esterna di un insieme, è logico pensare che, quando queste coincidono, abbiamo a che fare esattamente con la misura di quell'insieme, più o meno come l'area delle partizioni inferiori e superiori per Riemann.)\\

Risulta che $\mathcal{M}$ è una $\sigma$-algebra tale che $\mathcal{B}\subset \mathcal{M} \subset 2^{\mathbb{R}}$ e che  $\lambda(\cdot)$ è una misura, detta \textbf{misura di Lebesgue (unidimensionale)}.

\subsection{Misura di conteggio}
Indicato con $\# J$ il numero di elementi di un qualsiasi insieme finito $J$, fissiamo $ S\subseteq\Omega$. Allora, la funzione d'insieme $\gamma_{S}$ : $2^{\Omega}\mapsto[0,\ +\infty]$ così definita:

$$\gamma_{S}(A)=\left\{\begin{array}{ll}
\# A\cap S & \mathrm{s}\mathrm{e}\ A\cap S\ \mathrm{f}\mathrm{i}\mathrm{n}\mathrm{i}\mathrm{t}\mathrm{o}\\
+\infty & \mathrm{a}\mathrm{l}\mathrm{t}\mathrm{r}\mathrm{i}\mathrm{m}\mathrm{e}\mathrm{n}\mathrm{t}\mathrm{i}\ 
\end{array}\right.$$

che conta il numero di elementi comuni ad $A$ e $S$, è una misura, detta \textbf{ misura di conteggio indotta} da $S$ su $\Omega$.

\begin{thm}[dim] Proprietà della misura
\begin{itemize}
\item \textsc{additività}: $\displaystyle \mathrm{m}(\bigcup_{i=1}^{n}A_{i})=\sum_{i=1}^{n}\mathrm{m}(A_{i})$ , se $(A_{i})_{i\leq n}$ è disgiunta; 
\item $\mathrm{m}(A_{1} \cap A_{2})+\mathrm{m}(A_{1}\cup A_{2})=\mathrm{m}(A_{1})+\mathrm{m}(A_{2})$ ;
\item [$\star$]  $\mathrm{m}(A_{2}\backslash A_{1})=\mathrm{m}(A_{2})-\mathrm{m}(A_{1})$ , se $ A_{1}\subseteq A_{2}\mathrm{ \, e \,}\mathrm{m}(A_{1})<+\infty$; 
\item \textsc{monotonia} : $\mathrm{m}(A_{1})\leq \mathrm{m}(A_{2})$ , se $A_{1}\subseteq A_{2}$;
\item \textsc{subadditività}: $\displaystyle \mathrm{m}(\bigcup_{i\in I}A_{i})\leq\sum_{i\in I}\mathrm{m}(A_{i})$ , se $I$ è discreto;
\item \textsc{continuità dal basso}: $\displaystyle \mathrm{m}(A_{n})\uparrow \mathrm{m}(\bigcup_{n\geq 1}A_{n})$ , se $(A_{n})_{n\geq 1}$ è una successione non decrescente;
\item \textsc{continuità dall'alto}: $\displaystyle \mathrm{m}(A_{n})\downarrow \mathrm{m}(\bigcap_{n\geq 1}A_{n})$ , se $(A_{n})_{n\geq 1}$ è una successione non crescente tale che $\mathrm{m}(A_{m})<+\infty$ per qualche $m$;
\item \textsc{formula d'inclusione-esclusione}: sia $\mathrm{m}(A_{i})<+\infty$ $(i=1,\ \ldots,\ n)$. Allora, $ \displaystyle
\mathrm{m}\left(\bigcup_{i=1}^n A_{i}\right)=\sum_{\emptyset\subset J\subseteq\{1,...,n\}}(-1)^{\# J-1}\mathrm{m}\left(\bigcap_{j \in J} A_{j}\right) ;$\\

Ossia una cosa del tipo: $\left \bigcup_{{i=1}}^{n}A_{i}\right|=\sum _{{i=1}}^{n}\left|A_{i}\right|-\sum _{{1\leq i<j\leq n}}\left|A_{i}\cap A_{j}\right|+\sum _{{1\leq i<j<k\leq n}}\left|A_{i}\cap 
A_{j}\cap A_{k}\right|-\ \cdots \ (-1)^{{n-1}}\left|A_{1}\cap \cdots \cap A_{n}\right|$ senza valori assoluti e con la funzione misura
\item Sia $\mathrm{m}(A_{i})=0$ per ogni $i\in I$ con $I$ discreto. Allora, $\displaystyle \mathrm{m}(\bigcup_{i\in I}A_{i})=0$;
\item Sia $\mathrm{m}(\Omega)<+\infty$ e $\mathrm{m}(A_{i})=\mathrm{m}(\Omega)$ per ogni $i\in I$ con $I$ discreto. Allora, $ \mathrm{m}(\bigcap_{i\in I}A_{i})=\mathrm{m}(\Omega)$ .
\end{itemize}
\end{thm}

\subsection{Applicazioni misurabili}
\subsubsection{Misurabilità}
\begin{defn}
Dati due \textbf{spazi di misura} $(\Omega, \mathcal{A}) \, \mathrm{e} \, (\Omega', \mathcal{A'})$ una applicazione $f$ : $\Omega\mapsto\Omega'$ è $(\mathcal{A},\ \mathcal{A}')$-\textbf{misurabile} se $f^{-1}(A')\in \mathcal{A}$ per ogni $A'\in \mathcal{A}'$
\end{defn}

Vediamo ora due teoremi che affermano in breve che restringendo il dominio di una applicazione misurabile il risultato è ancora una applicazione misurabile; e che la composta fra due applicazioni misurabili è ancora una applicazione misurabile.

\begin{thm}[dim]
Sussistono le seguenti proposizioni:

$\bullet$ Siano $f$ un'applicazione $(\mathcal{A},\mathcal{A}')$-misurabile $\mathrm{e}\, S$ un sottoinsieme non vuoto di $\Omega$. Allora, considerata $\mathcal{A}\cap S$, traccia della $\sigma$-algrebra $\mathcal{A}$ su $S$, la restrizione $f|_{S}$ è un'applicazione $(\mathcal{A}\cap S, \mathcal{A}')$-misurabile;\\

$\bullet$ Dati uno spazio di misura ($\Omega, \mathcal{A}$), uno spazio di misura ($\Omega', \mathcal{A'}$) ed un ultimo spazio di misura ($\Omega'', \mathcal{A}''$), siano $f$ : $\Omega\rightarrow\Omega'$ $(\mathcal{A}, \mathcal{A'}$)- misurabile \, $\mathrm{e}$\, $g$ : $\Omega'\rightarrow\Omega''$ $(\mathcal{A}',\ \mathcal{A}'')$-misurabile. Allora, l'applicazione composta $g \circ f$ è $(\mathcal{A},\ \mathcal{A}'')$-misurabilie
\end{thm}

\begin{thm} [$\star$ dim ] \textbf{Criterio standard di misurabilità}. \, Dati due spazi di misura $(\Omega,\mathcal{A})$ e $(\Omega',\mathcal{A}')$
sia $\sigma(\mathcal{F}')=\mathcal{A}'$, sia cioè $\mathcal{F}'$ un insieme generatore di $\mathcal{A'}$. Allora $f$ : $\Omega\mapsto\Omega'$ è $(\mathcal{A}, \mathcal{A}')$-misurabile se $f^{-1}(F')\in \mathcal{A}$ per ogni $F'\in \mathcal{F}'.$ 
\end{thm}

Dice che non occorre verificare la misurabilità su ogni insieme di $\mathcal{A}$ (per fortuna! perché potrebbero essere tanti), ma basta verificarla sulla famiglia di i-fnsiemi che forma il generatore di $\mathcal{A}'$.\\


\noindent  
Le variabili aleatorie sono un caso particolare di funzioni misurabili.

\noindent
\begin{thm}
$\forall A' \in \mathcal{A}', f^{-1} \in \mathcal{A} \Leftrightarrow  \forall F' \in \mathcal{F}', f^{-1}(F') \in \mathcal{A} $
\end{thm}
\noindent
Cioè per dimostrare la misurabilità della funzione f basterà dimostrare la condizione definita su un sistema di generatori, e non su ogni elemento della $\mathcal{A}'$

\subsubsection{Borel-misurabilità}
\begin{defn}
Chiamiamo $\mathcal{A}$-\textbf{Borel misurabile} (in breve \textbf{Borel misurabile}) ogni funzione $(\mathcal{A},\mathcal{B})$-misurabile o $(\mathcal{A}, \mathcal{B}^{*})$-misurabile
\end{defn}


\subsubsection{Teor su funzioni Borel-misurabili}
\begin{thm} [dim solo per funzioni continue, dim parziale] Sussistono le seguenti proposizioni:

$\bullet$ Dati gli spazi misurabili $(\mathbb{R}^*,\mathcal{B^*)'}$ e $(\mathbb{R}^*,\mathcal{B^*)''}$, sia $\emptyset\neq S\in B^{*}'$. Allora le funzioni di $S$ in $\mathbb{R}^{*}''$ che hanno un numero discreto di  punti di discontinuità sono $(B^{*}'\cap S)$-Borel misurabili  ;\\

Un esempio sono le funzioni continue oppure le funzioni monotone.

$\bullet$ Dati gli spazi misurabili $(\Omega,\mathcal{A)}$, $(\mathbb{R}^*,\mathcal{B^*)}'$ e $(\mathbb{R}^*,\mathcal{B^*)}''$ siano $f$ : $\Omega\mapsto \mathbb{R}^{*}$ Borel misurabile e $S\in \mathcal{B}^{*}'$ tale che n$f(\Omega)\subseteq S$. Allora, qualunque sia la funzione $g$ : $S\mapsto \mathbb{R}^{*}''$ avente l'insieme dei punti di discontinuità discreto,  $g \circ f$ è Borel misurabile.
\end{thm}

Nota che non serve richiedere che $g$ sia borel-misurabile, perché scatta il teorema precedente.


\begin{thm} Siano $f, g, f_{n}$ funzioni Borel misurabili. Allora, dopo aver denotato compattamente con “$\{f=g\}"=\{\omega| f(\omega)=g(\omega)\}$:\\

$\bullet\{f=g\}, \{f<g\}, \{f>g\}, \{f\leq g\} e \{f\geq g\}\in \mathcal{A}$;

$\bullet$ Sono Borel misurabili le funzioni
$$
\inf_{n\geq 1}f_{n}\ \sup_{n\geq 1}f_{n}\ \lim_{n\rightarrow +\infty }\inf_{} f_{n}\ \lim_{n\rightarrow +\infty }\sup_{} f_{n}\ \lim_{n\rightarrow+\infty}f_{n}
$$

$\bullet$ Sono Borel misurabili le funzioni $f$ e $g$, se definite ovunque, $f+g\displaystyle$ e $\frac{f}{g}$; 

$\bullet$ Sia $f$ derivabile. Allora $f'$ è Borel misurabile;

$\bullet$ Sia $f_{n}\geq 0$. Allora, la serie $\displaystyle \sum_{n\geq 1}f_{n}$ è una funzione Borel misurabile.
\end{thm}

\subsubsection{Funzione semplice}
\begin{defn}
$f$ : $\Omega\rightarrow \mathbb{R}^{*}$ è una $\mathbf{\mathcal{A}}$-\textbf{funzione semplice} (in breve \textbf{funzione semplice}) se è $\mathcal{A}$-Borel misurabile $\mathrm{e}$ l'insieme-immagine $f(\Omega)$ è finito
\end{defn}

\subsubsection{Funzione indicatrice}
\begin{thm}[dim rapida] La funzione indicatrice di $S \subset \Omega$ $I_{s}(\omega)$ è una funzione $(\mathcal{A}, \mathcal{B})$- misurabile se e solo se $S\in \mathcal{A}$
\end{thm}

\begin{thm} Se $S\in \mathcal{A}$, la {funzione indicatrice} di $ S\subset\Omega$:
$I_{S}(\omega)=\left\{\begin{array}{ll}
1 & \mathrm{s}\mathrm{e}\ \omega\in S\\
0 & \mathrm{s}\mathrm{e}\ \omega\not\in S
\end{array}\right.$
è una funzione semplice 
\end{thm}

\subsubsection{Proprietà della funzione indicatrice}
\begin{thm}[dim] Valgono le seguenti proprietà:
\begin{itemize}
    \item  $I_{S^{c}}=1-I_{S}$
    \item $I_{S\cap T}=I_{S}\cdot I_{T}$
    \item $I_{S\cup T}=I_{S}+I_{T}-I_{S\cap T}$
\end{itemize}
Data $f$ semplice (qui $(\mathcal{A,B}^*)$-misurabile) e posto per convezione “$\{f=y\}$’’ $=f^{-1}(\{y\})$ per ogni $y \in \mathbb{R}^{*}$:

\begin{itemize}
    \item $\left(\{f=y\}\right) _{y\in f(\Omega)}$ è una partizione finita di $\Omega$ formata da elementi di $\mathcal{A}$ 
    \item $f=\displaystyle \sum_{y\in f(\Omega)}yI_{\{f=y\}}$
\end{itemize}
\end{thm}
\noindent
La prima: in pratica raggruppo gli $\omega$ in base alla loro immagine

\noindent
La seconda: è una combinazione lineare con pesi i valori $y_1,...,y_n$. Esplicitamente è: $y_1 \cdot I_{\{f=y_1 \}}
(\omega)+...+y_n \cdot  
I_{\{f=y_n\}} 
(\omega)$, fino $n$ perché funzione semplice



\subsubsection{Lemma fondamentale (funzioni semplici)}
\begin{thm}  \textbf{Lemma fondamentale} Sussistono le seguenti proposizioni:

$\bullet$ Somme, quozienti (se definiti ovunque) e prodotti di funzioni semplici sono ancora funzioni semplici;

$\star \star$ Sia $f$ una funzione Borel misurabile non negativa. Esiste allora una successione $(f_{n})_{n\geq 1}$ di funzioni semplici, non negative, e a valori finiti tale che $f_{n}\uparrow f$;

$\bullet$ Sia $f$ una funzione Borel misurabile. Esiste allora una successione $(f_{n})_{n\geq 1}$ di funzioni semplici a valori finiti tale che $f_{n}\rightarrow f$ e $|f_{n}|\leq|f|$ per ogni $n.$
\end{thm}

\subsection{Integrale di Lebesgue}
$f$ : $\Omega\rightarrow \mathbb{R}^{*}$ funzione $\mathcal{A}$-Borel misurabile e sia $m$ una misura su $\mathcal{A}$. Definiamo l'integrale di Lebesgue per passi, di seguito i primi due mentre il terzo e ultimo passo avverrà in seguito:\\

\subssubsection{Integrale di funzioni semplici non negative}
\noindent
$1)$ \textbf{Integrale di funzioni semplici non negative}

Sia $f\geq 0$ semplice. Allora:
$$
\int_{\Omega}fd\mathrm{m}=\sum_{y\in f(\Omega)}y \cdot \mathrm{m}(\{f=y\})\geq 0
$$
In particolare, $\displaystyle \mathrm{m}(A)=\int_{\Omega}I_{A}d\mathrm{m}$ per ogni $A\in \mathcal{A}.$

\subsubsection{Integrale di  funzioni Borel misurabili non negative}
\noindent
$2)$ \textbf{Integrale di  funzioni Borel misurabili non negative}

Sia $f\geq 0$. Allora:\\

$\displaystyle \int_{\Omega}fd\mathrm{m}=\sup\{\int_{\Omega}g\, d\mathrm{m}$ : $0\leq g\leq f\; \mathrm{e}\;g$ funzione semplice \}$\geq 0$\\

\noindent
Il prossimo risultato consente il passaggio del limite sotto il segno d'integrale nel caso di successioni non decrescenti di funzioni Borel misurabili non negative.
\subsubsection{Teor della convergenza monotona}
\begin{thm} \textbf{Teorema della convergenza monotona}
Sia $(f_{n})_{n\geq 1}$ una successione di funzioni Borel misurabili non negative tale che $f_{n}\uparrow f$. Allora,
$$
\int_{\Omega}f_{n}d\mathrm{m}\uparrow\int_{\Omega}fd\mathrm{m}
$$
\end{thm}
\noindent
(ricordo che  $f_n$ è una successione non decrescente)\\

\noindent
In sostanza la tesi del secondo punto del Lemma fondamentale diventa l'ipotesi del Teor qui sopra:\\ successione di funzioni semplici, non negative : $f_n \uparrow f$ con $f$ B-mis $\Rightarrow$ \\ successione di funzioni B-mis, non negative : $g_n \uparrow g$ con g ($\geq 0$, B-mis) \\




\noindent
Conseguentemente, per il lemma fondamentale, \textbf{l'integrale di una funzione Borel misurabile non negativa è il limite di una opportuna successione non decrescente di integrali di funzioni semplici non negative a valori finiti}.\\

\subsubsection{Funzioni Borel misurabili qualsiasi}
\subsubsection{Parte positiva e negativa di una funzione}
Le funzioni:
\begin{itemize}
    \item [] \textbf{parte positiva}: $f^{+}=\displaystyle \max(0,\ f)$ 
    \item [] \textbf{parte negativa}: $f^{-}=\displaystyle \max(0,\ -f)$
\end{itemize}

\noindent
sono Borel misurabili in quanto trasformate continue di $f$.

\begin{thm} Vale inoltre che 
$$
f=f^{+}-f^{-}
$$
$(fg)^{+}=\left\{\begin{array}{ll}
fg^{+} & \mathrm{s}\mathrm{e}\ f\geq 0\\
fg^{-} & \mathrm{s}\mathrm{e}\ f\leq 0
\end{array}\right.$

$$
|f|=f^{+}+f^{-}
$$
$(fg)^{-}=\left\{\begin{array}{ll}
fg^{-} & \mathrm{s}\mathrm{e}\ f\geq 0\\
-fg^{+} & \mathrm{s}\mathrm{e}\ f\leq 0
\end{array}\right.$
\end{thm}

\noindent
$3)$ \textbf{Integrale di una funzione $f$  qualsiasi} \\
\noindent
Sia $f$ qualsiasi. 

\subsubsection{Funzione sommabile}
Considerati gli integrali $\displaystyle \int_{\Omega}f^{+}d\mathrm{m}, \displaystyle \int_{\Omega}f^{-}d\mathrm{m}$ e  supposto che {non siano entrambi infiniti} (ocio), chiamiamo \textbf{$\mathrm{m}$-sommabile} (in breve \textbf{sommabile}) la funzione $f$ e poniamo:
$$
\int_{\Omega}fd\mathrm{m}=\int_{\Omega}f^{+}d\mathrm{m}-\int_{\Omega}f^{-}d\mathrm{m}
$$
Il valore dell'integrale di una funzione sommabile $f$ è quindi:

$\bullet+\infty$, se $\displaystyle \int_{\Omega}f^{+}d\mathrm{m}=+\infty$

$\bullet-\infty$, se $\displaystyle \int_{\Omega}f^{-}d\mathrm{m}=+\infty$

$\bullet$ finito, se $\displaystyle \int_{\Omega}f^{+}d\mathrm{m}$  e $\displaystyle  \int_{\Omega}f^{-}d\mathrm{m}$ sono entrambi finiti.\\

\subsubsection{Funzione integrabile}
\noindent
In quest'ultimo caso diremo che $f$ è \textbf{m-integrabile} (in breve \textbf{integrabile}). Ovvero che  $\displaystyle 
\int_{\Omega}fd\mathrm{m}$ è finito.\\

\noindent
Sia ora $A\in \mathcal{A}$. Allora $fI_{A}$ è una funzione Borel misurabile; inoltre, tenuto conto delle disuguaglianze
$$
(fI_{A})^{+}=f^{+}I_{A}\leq f^{+} 
$$
e
$$
(fI_{A})^{-}=f^{-}I_{A}\leq f^{-}
$$
 $fI_{A}$ risulta sommabile (integrabile) se $f$ è sommabile (integrabile). Conseguentemente, nel caso di sommabilità della funzione $f$, poniamo:
$$
\int_{A}fd\mathrm{m}=\int_{\Omega}fI_{A}d\mathrm{m}
$$
e chiamiamo “${\displaystyle \int_{A}fd\text{m}}$" \textbf{l'integrale di Lebesgue di $f$ su $A$.}

\subsection{Proprietà dell'integrale di Lebesgue}
\begin{thm} [Proprietà dell'integrale di Lebesgue] Sussistono le seguenti proposizioni:\\
\begin{itemize}
\subsubsection{Bilinearità}
\item \textsc{linearità rispetto alla funzione}: Siano $f, g$ funzioni sommabili e $ \alpha, \beta$ numeri reali tali che $\alpha f+\beta g$ risulti definita ovunque. Allora,
$$
\int_{\Omega}(\alpha f+\beta g)d\mathrm{m}=\alpha\int_{\Omega}fd\mathrm{m}+\beta\int_{\Omega}gd\mathrm{m}
$$
ogni qual volta non siano infiniti di segno opposto gli addendi che compaiono al secondo membro dell'uguaglianza;

\item (dim) \textsc{linearità rispetto alla misura}: Siano m una misura su $\mathcal{A}$ e $f$ una funzione $\mathrm{m}$-sommabile e $\mathrm{m'}$-sommabile. Allora, qualunque siano $\mathrm{i}$ numeri reali non negativi $\alpha$ e $\alpha'$, si ha
$$
\int_{\Omega}fd(\alpha \mathrm{m}+\alpha'\mathrm{m}')=\alpha\int_{\Omega}fd\mathrm{m}+\alpha'\int_{\Omega}fd\mathrm{m}'
$$
ogni qual volta non siano infiniti di segno opposto gli integrali che compaiono al secondo membro dell'uguaglianza;

\subsubsection{Monotonia}
\item [$\bullet$] \textsc{monotonia} Siano $f, g$ funzioni sommabili tali che $f\leq g$. Allora
$$
\int_{\Omega}f\, d\mathrm{m}\leq\int_{\Omega}g \, d\mathrm{m}
$$
\subsubsection{Additività}
\item  \textsc{additività}: Sia $f$ una funzione sommabile. Allora presi due  insiemi disgiunti qualsiasi $A_{1}\mathrm{\;e\;}A_{2}$, risulta
$$
\int_{A_{1}\cup A_{2}}fd\mathrm{m}=\int_{A_{1}}fd\mathrm{m}+\int_{A_{2}}fd\mathrm{m}
$$


\item \textsc{monotonia} Siano $f, g$ funzioni sommabili tali che $f\leq g$. Allora
$$
\int_{\Omega}f \, d\mathrm{m}\leq\int_{\Omega}g \, d\mathrm{m}
$$
\subsubsection{Annullamento sugli insiemi di misura nulla}
\item [$\star$ ](dim) \textsc{annullamento sugli insiemi di misura nulla} : Sia $f$ una funzione sommabile e $\mathrm{m}(A)=0$. Allora
$$
\int_{A}f \, d\mathrm{m}=0
$$
\end{itemize}
\end{thm}

\subsection{Teor su funzioni sommabili e integrabili}
\begin{thm} Siano $f,g$ due funzioni Borel misurabili. Allora:
\begin{itemize}
\item (dim) $f$ è sommabile se:\\
$\star$ $f \geq g$ con $g$ funzione sommabile t.c. $\int_{\Omega} g dm > -\infty$\\ 
- $f \leq g$ con $g$ funzione sommabile t.c. $\int_{\Omega}g dm<+\infty$

\item (dim) $f$ è integrabile se e solo se lo è il suo valore assoluto;
\item (dim) $f$ è integrabile se $|f|\leq g$ con $g$ funzione integrabile;
\item [$\star$] (dim) $f \cdot g$ è integrabile se $f^{2}, g^{2}$ sono integrabili;
\end{itemize}
\end{thm}

\subsection{Insieme trascurabile}
\begin{defn}
Chiamiamo \textbf{m-trascurabile} (in breve \textbf{trascurabile}) ogni sottoinsieme $S$ di $\Omega$ per il quale esista un insieme $A\supseteq S$ tale che $\mathrm{m}(A)=0.$
\end{defn}

\noindent
Sono trascurabili, oltre all'insieme vuoto, i sottoinsiemi di un insieme trascurabile e l'unione di una famiglia discreta di insiemi trascurabili.\\ 

\noindent Inoltre, per la monotonia della misura, un elemento di $\mathcal{A}$ è trascurabile se e solo se è di misura nulla.

\subsection{Proprietà m-quasi ovunque}
\begin{defn} Diremo che una proprietà $P$ o una relazione binaria $\asymp$ riguardanti applicazioni di dominio $\Omega$, sussiste m\textbf{-quasi ovunque} (in breve \textbf{quasi ovunque}) se il suo campo di validità include il complementare di un insieme trascurabile. Per indicare tale situazione, useremo, rispettivamente, le notazioni $P$ (m-q.o.) e $\asymp$ (m-q.o.)
\end{defn}

\noindent
Oss. Non particolarmente utile:

$-\{f\}$ finita (m-q.o.), se l'insieme $\{|f|=+\infty\}$ è trascurabile

$-\{f\leq g\}$ (m-q.o.), se l'insieme $\{f>g\}$ è trascurabile

$-\{f<g\}$ (m-q.o.), se l'insieme $\{f\geq g\}$ è trascurabile

$-\{f=g\}$ (m-q.o.), se l'insieme $\{f\neq g\}$ è trascurabile

$-\{\displaystyle \lim_{n\rightarrow+\infty}f_{n}\}$ (m-q.o.), se l'insieme $A_{\lim}^{c}$ è trascurabile;

$-\{f_{n}\rightarrow f\}$ (m-q.o.), se l'insieme $\{f_{n}\rightarrow f\}^{c}$ è trascurabile;

$-\{f_{n}\uparrow f\}$ (m-q.o.), se l'insieme $\{\omega\in\Omega\ :\ \neg(f_{n}(\omega)\uparrow f(\omega))\}$ è trascurabile.\\

In pratica se una funzione è definita su un insieme di $\omega$ trascurabile, allora è definita m-quasi ovunque sul complementare di quell'insieme di $\omega$.   

\vspace{3mm}

\subsection{Teor su integrali di Lebesgue}
\begin{thm}[dim] Sia $f$ una funzione sommabile. Allora:
\begin{itemize}
\item  $\displaystyle \int_{\Omega}g \, d\mathrm{m}\leq\int_{\Omega}fd\mathrm{m}$, se $g\leq f$ (m-q.o.) e $g$ è sommabile

\item  $\displaystyle \int_{\Omega}g \, d\mathrm{m}=\int_{\Omega}fd\mathrm{m}$, se $f=g$ (m-q.o.) e $g$ è Borel-misurabile (ocio)

\item  $\displaystyle \left| \int_{\Omega}fd \mathrm{m}\right| \leq \int_{\Omega} \left| f \right| d\mathrm{m}$

\item  $\displaystyle \mathrm{m}(A)\inf f(A)\leq\int_{A}fd\mathrm{m}\leq \mathrm{m}(A)\sup f(A)$ per ogni $A$ \\ (come teor media integrale: b*h=area)

\item[$\star$]  Se $\displaystyle f\geq 0\mathrm{\;e\;}\int_{\Omega}fd\mathrm{m}=0$, allora $f=0$ (m-q.o.)

\item[$\star \star$]  Se $\mathrm{m}(A)>0\mathrm{\;e\;}f(\omega)>0$ per ogni $\omega\in A$, allora $\displaystyle \int_{A}fd\mathrm{m}>0$

\item[$\star \star \star $]  Se $f$ è integrabile, allora $f$ è finita quasi ovunque.
\end{itemize}
\end{thm}


\subsection{Ulteriori proprietà di convergenza}
\subsubsection{Teor della convergenza dominata}
\begin{thm} \textbf{Teorema della convergenza dominata}
Sia $(f_{n})_{n\geq 1}$ una successione di funzioni Borel misurabili tale che $f_{n}\rightarrow f$. Inoltre, sia $g$ una funzione integrabile tale che $|f_{n}|\leq g$ (m-q.o.) per ogni $n$. Allora  $f$ ed anche $(f_{n})_{n\geq 1}$ sono integrabili $\mathrm{e}$ si ha
$$
\int_{\Omega}f_{n} \, d\mathrm{m}\rightarrow\int_{\Omega}f \, d\mathrm{m}
$$
\end{thm}

\subsubsection{Teor d'integrazione per serie}
\begin{thm} \textbf{Teorema d'integrazione per serie}
Siano $(f_{n})_{n\geq 1}$ una successione di funzioni Borel misurabili non negative $\mathrm{e}$ $(\alpha_{n})_{n\geq 1}$ una successione di numeri reali non negativi. Allora,
$$
\int_{\Omega}\left(\sum_{n\geq 1}\alpha_{n}f_{n} \right) d\mathrm{m}=\sum_{n\geq 1}\alpha_{n} \left( \int_{\Omega}f_{n}d\mathrm{m}\right)
$$
\end{thm}
Questo teorema ha riunificato i casi di v.a. a valori finiti, infinitamente numerabili e assolutamente continue

\subsubsection{Legame fra integrale di Lebesgue e serie numeriche}
\begin{oss} \textbf{Integrale di Lebesgue e serie numeriche} Sia $ S\subseteq\Omega$ un insieme discreto avente $ N\leq+\infty$ elementi $\mathrm{e}$ sia $f$ : $\Omega\mapsto \mathbb{R}^{*}$ $\mathrm{u}\mathrm{n}\mathrm{a}$ funzione $\gamma_{S}$-sommabile. Considerata allora una numerazione $s_{1}, s_{2}, \ldots$ di $S$, riesce
$$
\displaystyle \int_{\Omega}fd\gamma_{S}=\sum_{i=1}^{N} f^{+}(s_{i})-\sum_{i=1}^{N} f^{-}(s_{i})=\sum_{i=1}^{N} f^{+}(s_{i})+\sum_{i=1}^{N}(-f^{-}(s_{i})) 
$$
quindi, se $ N=+\infty$, l'integrale pu\`{o} essere calcolato sommando le serie dei valori, rispettivamente, positivi $\mathrm{e}$ negativi della restrizione $f|_{S}.$\\ 

\noindent
Sussiste l'uguaglianza $\displaystyle \int_{\Omega}fd\gamma_{S}=\sum_{n\geq 1}f(S)$ ogniqualvolta la serie numerica $\displaystyle \sum_{n\geq 1}f(s_{n})$ \`{e} permutabile, come avviene quando $f$ \`{e} di segno costante (serie a termini positivi) oppure \`{e} $\gamma_{S}$-integrabile (serie assolutamente convergente).
\end{oss}

\newpage


%==================================
%DISPENSA 2
%===================================

\section{PROBABILITA' 1}
\subsection{Eventi, variabili aleatorie, probabilità, enti aleatori}
Dati 
\begin{itemize}
    \item $\Omega\neq\emptyset$ (partizione dell'evento certo)
    \item $\mathcal{A}$ $\sigma$-algebra su $\Omega$ (degli eventi di interesse)
\end{itemize}


\begin{defn} Chiamiamo:
\begin{itemize}
    \item  \textbf{caso elementare} (di $\Omega$) ogni elemento di $\Omega$
    \item \textbf{evento} (di $\Omega$) ogni elemento di $\mathcal{A}$
    \item \textbf{probabilità} (sugli eventi di $\Omega$) ogni misura $\mathrm{P}$ su $A$ tale che $\mathrm{P}(\Omega)=1$;
    \item \textbf{variabile aleatoria} (su $\Omega$) ogni funzione $\mathcal{A}$-Borel misurabile a valori nella retta reale;
    \item \textbf{variabile aleatoria estesa} (su $\Omega$) ogni funzione $\mathcal{A}$-Borel misurabile a valori nella retta reale ampliata;
\end{itemize}
\end{defn}
\subsubsection{Ente aleatorio}

\noindent
Considerato lo spazio di misura$(\Omega, \mathcal{A})$ ed inoltre un altro spazio di misura ($\mathfrak{X}, \aleph$), chiamiamo \textbf{ente aleatorio (su $\Omega$)} a valori in $\mathfrak{X}$ ogni applicazione di $\Omega$ in $\mathfrak{X}$ che sia $(\mathcal{A}, \aleph)$-misurabile.


\subsubsection{Legge di un ente aleatorio}
Dato un ente aleatorio $X$ a valori in $\mathfrak{X}$ e  $A \in \aleph$, consideriamo l'applicazione $\mathrm{P}_{X}$ : $\mathcal{\aleph}\rightarrow[0,1]$:
$$
\mathrm{P}_{X}(\mathrm{A})=\mathrm{P}(\{X\in \mathrm{A}\})\ =\mathrm{P}(\{\omega|X(\omega)\in \mathrm{A}\}).
$$
che è una probabilità su $\aleph$.





\begin{defn}
La probabilità $\mathrm{P}_{X}$ si chiama \textbf{legge} (o \textbf{distribuzione}) dell'ente aleatorio $X$. 
\end{defn}

\noindent
Essa è la ‘probabilità immagine' di $P$ mediante $X$. \\

\noindent
$(P \circ X)(A)=P(X \in A)=P(X^{-1}(A))=P_X(A)=P_X(A)$\\

La legge di $X$ associa ad ogni A $\in \aleph$ la probabilità $\mathrm{P}(X\in \mathrm{A})$ che l'ente aleatorio $X$ assuma, come determinazione, un elemento dell'insieme A.\\


\noindent
$\Omega \;\; \overset{X}{\longrightarrow} \;\; \mathfrak{X}\\
\mathcal{A} \;\;\;\;\;\;\; \,\,\,\;\;\; \aleph_A  \,\,\,\;\;\;$
Ocio che $\{X \in \mathcal{A}\} \in \mathcal{A}$, mentre $A \in \aleph \\
P \;\;\;\;\;\;\;\;\;\;\; P_{X}$
\\

\begin{thm}[dim x casa] Risulta:\\

$\bullet F_{X}(x)\leq F_{X}(x')$ , se $x\leq x'$;

$\bullet F_{X}(-\displaystyle \infty)=\lim_{x\rightarrow-\infty}F_{X}(x)=0$;

$\bullet F_{X}(+\displaystyle \infty)=\lim_{x\rightarrow+\infty}F_{X}(x)=1$;

$\bullet F_{X}(x^{+})=\displaystyle \lim_{t\rightarrow x^{+}}F_{X}(t)=F_{X}(x)$ ;

$\bullet F_{X}(x^{-})=\displaystyle \lim_{t\rightarrow x^{-}}F_{X}(t)=\mathrm{P}(X<x)=\mathrm{P}_{X}(]-\infty,\ x[)$ ;

$\bullet F_{X}(x^{+})-F_{X}(x^{-})=\mathrm{P}(X=x)$ .

$\star$ Inoltre, se $\mathrm{m}'$ è una misura su $\mathcal{B}$ tale che $\mathrm{m} ]-\infty, x$]) $=F_{X}(x)$ per ogni numero reale $x$, si ha $\mathrm{m}'=\mathrm{P}_{X}.$
\end{thm}

\subsubsection{Ente aleatorio trasformato}
\begin{thm} [dim]
Sia $g:\mathrm{X}\mapsto \mathbb{R}^{*}$ una funzione sommabile rispetto alla legge di $X.$ Riesce allora
$$
\int_{\{X\in \mathrm{A}\}}g(X)d\mathrm{P}=\int_{\mathrm{A}}g \, d\mathrm{P}_{X}
$$
per ogni $\mathrm{A}\in\aleph.$
\end{thm}


\subsubsection{Funzione di densità di un ente aleatorio}
Posso unificare i 2 seguenti casi poiché la Legge è ricostruibile via integrazione.

\begin{defn}  \textsc{variabili aleatorie discrete} V.a. $X$ con rango $B=\{x_{1},\ x_{2},\ \ldots\}$

$f(x)=\mathrm{P}(X=x)\geq 0$ (funzione di probabilità)\\
\noindent
Considerata la misura di conteggio $\gamma_{B}$ su $\aleph=2^{\mathbb{R}}$, otteniamo
$$
\mathrm{P}_{X}(\mathrm{A})\ =\mathrm{P}(\bigcup_{n\in\{n:x_{n}\in \mathrm{A}\}}\{X=x_{n}\})=\sum_{n\in\{n:x_{n}\in \mathrm{A}\}}f(x_{n})
$$
$$
=\ \sum_{n\geq 1}f(x_{n})I_{\mathrm{A}}(x_{n})=\int_{\mathrm{A}}fd\gamma_{B}
$$
per ogni $\mathrm{A}\in\aleph.$\\

\end{defn}

\begin{defn} \textsc{Variabili aleatorie assolutamente continue} V.a. assolutamente continua $X$ con $f$ funzione di densità, cioè $f\geq 0$ e
$$
F_{X}(x)=\int_{-\infty}^{x}f(t)dt
$$
per ogni numero reale $x$. Allora
$$
\mathrm{P}_{X}(B)=\int_{B}f(t)dt
$$
per ogni boreliano $B.$
\end{defn}

\noindent
Nei due casi classici considerati la legge $P_X$ può essere determinata via integrazione di una funzione $f\geq 0$ rispetto ad una misura di riferimento (nel primo caso $\gamma_B$; nel secondo $\lambda$).\\


\noindent
Ritornando al caso generale, supponiamo che sulla $\sigma$-algebra $\aleph$ sia definita una\textbf{ misura di riferimento} $\mu.$

\subsubsection{Mu-densità di un ente aleatorio}
\begin{defn} Chiamiamo \textbf{$\mu$-\textbf{densità} di $X$} ogni funzione $f$ :  $\mathbb{R} \mapsto[0,\ +\infty]$  $\aleph$-Borel misurabile tale che
$$
\mathrm{P}_{X}(\mathrm{A})=\int_{\mathrm{A}}fd\mu
$$
per ogni $\mathrm{A}\in\aleph$
\end{defn}




\subsubsection{Non unicità della densità di un ente}
\begin{thm} [dim] Siano $f,\tilde{f}$ due $\mu$-densità di $X$. Allora,
$$
\tilde{f}=f\ (\mu -q.o.)\ 
$$
\end{thm}

\noindent
Una $\mu$-densità è definita ($\mu-$q.o.), \textbf{a meno di insiemi $\mu$-trascurabili} quindi non è unica.


\subsubsection{Collegamento fra densità e legge di un ente}
\begin{thm} [dim.] Sia $f$ una $\mu$-densità di $X$. Se $g: \mathfrak{X}\rightarrow \mathbb{R}^*$ è sommabile rispetto alla legge di $X$, allora
$$
\int_{\mathrm{A'}}g \,d\mathrm{P}_{X}=\int_{\mathrm{A'}}gfd\mu
$$
per ogni $A'\in \aleph$.
\end{thm}

\subsubsection{Teor fondamentale del calcolo delle probabilità}
\begin{thm} [dim] \textbf{Teorema fondamentale del calcolo delle probabilità} 
Sia $g:\mathfrak{X}\mapsto \mathbb{R}^{*}$ una funzione sommabile rispetto alla legge di $X$. Riesce allora

$\displaystyle \int_{\{X\in \mathrm{A}\}}g(X)d\mathrm{P}=\left\{\begin{array}{ll}
\displaystyle \int_{\mathrm{A}}g \, d\mathrm{P}_{X} & \\
 & \\
\displaystyle \int_{\mathrm{A}}g \; fd\mu & \mathrm{s}\mathrm{e}\ f\ \text{é}\ \mathrm{u}\mathrm{n}\mathrm{a}\ 
\mu \mathrm{-}\mathrm{d}\mathrm{e}\mathrm{n}\mathrm{s}\mathrm{i} \mathrm{tà}\ \mathrm{d}\mathrm{i}\ X
\end{array}\right.$
per ogni $\mathrm{A}\in\aleph.$
\end{thm}




\subsubsection{Enti aleatori equidistribuiti}
\begin{defn}
Due enti aleatori $X, Y$ a valori in $\mathfrak{X}$ si dicono \textbf{equidistribuiti} se $\mathrm{P}_{X}=\mathrm{P}_{Y},$ cioè se hanno la medesima legge.
\end{defn}

\begin{thm} [dim] Conseguentemente, $X$ e $Y$ sono equidistribuiti se sono
uguali \textbf{quasi certamente} (cioè $\mathrm{P}$-quasi ovunque).
\end{thm}



\subsection{Speranza matematica di una variabile aletaoria}
Mentre prima..

\begin{itemize}
    \item $\mathcal{P}$ partizione dell'evento certo
\item $\mathcal{F} \sigma$-algebra di eventi logicamente dipendenti da $\mathcal{P}$
\item $\mathrm{P}\mathrm{r}$ probabilit\`{a} su $\mathcal{F}$
\item  $X:\mathcal{P}\rightarrow \mathbb{R}$ variabile aleatoria (cio\`{e} una funzione $\mathcal{F}$-Borel misurabile)
\item $\mathrm{E}(\mathrm{X})$ speranza matematica  è pari a:
\begin{itemize}
    \item  se $X$ ha rango  $B=\{x_{1}, ... ,\ x_{n}\}$ risulta
$\mathrm{E}(X)=x_{1}\mathrm{P}\mathrm{r}(X=x_{1})+\cdots+x_{n}\mathrm{P}\mathrm{r}(X=x_{n})$ $=\displaystyle \int_{\mathcal{P}}Xd\mathrm{P}\mathrm{r}$
\item  se $X$ è assolutamente continua: denotata con $f$ una sua funzione di densit\`{a}, dal Teorema fondamentale (ponendo $g(x)=x$) otteniamo
$\mathrm{E}(X)=\displaystyle \int_{-\infty}^{+\infty}xf(x)dx=\displaystyle \int_{\mathbb{R}}xf(x)\lambda(dx)=\displaystyle \int_{\mathcal{P}}Xd\mathrm{P}\mathrm{r}$
\end{itemize}
\end{itemize}

\noindent
Ora invece..\\

\begin{defn}
La \textbf{speranza matematica} (o \textbf{valore medio}) di una v.a. estesa P-sommabile $X$ è l'elemento della retta reale ampliata: 
$\displaystyle E(X)= \int_{\Omega} X d \mathrm{P}$
\end{defn}

\noindent
Nelle ipotesi del teorema fondamentale otteniamo \\

\noindent
$\displaystyle \mathrm{E}(g(X))=\displaystyle \int_{\Omega}g(X)d\mathrm{P}=\left\{\begin{array}{ll}
\displaystyle \int_{\mathbb{R}}g\, d\mathrm{P}_{X} & \\
 & \\
\displaystyle \int_{\mathbb{R}}gfd\mu & \mathrm{s}\mathrm{e}\ f\ \text{\`{e}}\ \mathrm{u}\mathrm{n}\mathrm{a}\ \mu- \mathrm{d}\mathrm{e}\mathrm{n}\mathrm{s}\mathrm{i}\mathrm{t}\text{\`{a}}\ \mathrm{d}\mathrm{i}\ X
\end{array}\right.$\\

\noindent
Nell'ulteriore caso particolare che $g$ sia l'identit\`{a} si ha \\

\noindent
$\mathrm{E}(X)=\left\{\begin{array}{ll}
\displaystyle \int_{\mathbb{R}}x \,\mathrm{P}_{X}\ (dx) & \\
 & \\
\displaystyle \int_{\mathbb{R}}\ x f(x)\mu(dx) & \mathrm{s}\mathrm{e}\ f\ \text{\`{e}}\ \mathrm{u}\mathrm{n}\mathrm{a}\ \mu- \mathrm{d}\mathrm{e}\mathrm{n}\mathrm{s}\mathrm{i}\mathrm{t}\text{\`{a}}\ \mathrm{d}\mathrm{i}\ X
\end{array}\right.$

\subsection{Proprietà}
\begin{thm}[dim solo se c'è $\star$] Riesce $\mathrm{E}(I_{A})=\mathrm{P}(A)$ e $\mathrm{E}(\alpha I_{\Omega})=\alpha$ per ogni $ A\in \mathcal{A}$ e $\alpha\in \mathbb{R}^{*}$. Inoltre, supposto che $X, Y$ siano $\mathrm{v}.\mathrm{a}$. estese con speranza matematica, risulta:
\begin{itemize} 
\subsubsection{Linearità}
\item \textsc{linearità} : Se $\alpha, \beta\in \mathbb{R}^{*}$ sono tali che $\alpha X+\beta Y$ \`{e} definita ovunque, allora
$$
\mathrm{E}(\alpha X+\beta Y)=\alpha \mathrm{E}(X)+\beta \mathrm{E}(Y)
$$
se $\alpha \mathrm{E}(X)$ e $\mathrm{E}(Y)$ non sono infiniti di segno opposto;
\subsubsection{Uguaglianza}
\item  se $X=Y$ $(\mathrm{P}-\mathrm{q}.\mathrm{c}.)$ allora $\mathrm{E}(X)=\mathrm{E}(Y)$ ;
\subsubsection{Monotonia}
\item \textsc{monotonia}: se $X\leq Y$ $(\mathrm{P}-\mathrm{q}.\mathrm{c}.)$ allora $\mathrm{E}(X)\leq \mathrm{E}(Y)$
\subsubsection{Valore assoluto}
\item $\left|\mathrm{E}(X)\right|\leq \mathrm{E}(|X|)$ ;
\subsubsection{Internalità}
\item \textsc{internalità}: $\displaystyle \mathrm{i}\mathrm{n}\mathrm{f}X\leq \mathrm{E}(X)\leq\sup X$;
\subsubsection{Internalità stretta}
\item [$\star$] \textsc{internalità stretta} se $\alpha<X(\omega)<\beta$ $\forall \omega$ allora $\alpha<\mathrm{E}(X)<\beta$;
\subsubsection{Disuguaglianza di Markov}
\item [$\star \star$] Se $X\geq 0$ e $a>0$, allora
$\mathrm{P}\mathrm{r}(X\geq a)\leq \displaystyle \frac{\mathrm{E}(X)}{a}$ disuguaglianza di Markov; 
\subsubsection{Ammissione }
\item Infine, una $\mathrm{v}.\mathrm{a}$. $Z$ ammette speranza matematica se $Z\geq X$ e $\mathrm{E}(X)>-\infty$ o $Z\leq X$ e $\mathrm{E}(X)<+\infty.$
\end{itemize}
\end{thm}

\begin{defn} Dato un  intervallo aperto $J$ (limitato o no) della retta reale\\
diremo che $g: J\mapsto \mathbb{R}$ è \textbf{convessa} se, dati due punti arbitrari $a, b\in J$ tali che $a<b,$ il grafico di $g_{[a,b]}$ sta sotto la retta passante per i punti $(a,\ g(a))$, $(b,\ g(b))$. Formalmente
$$
g(x)\leq\frac{g(b)-g(a)}{b-a}x+\frac{g(a)b-g(b)a}{b-a}
$$
per ogni $x\in[a,\ b].$
\end{defn}

\begin{oss}
Osservato che, dato $x\in[a,\ b]$, risulta
$$
x=\frac{b-x}{b-a}a+\frac{x-a}{b-a}b
$$
con
$$
\frac{b-x}{b-a}+\frac{x-a}{b-a}=1,\ \frac{b-x}{b-a},\ \frac{x-a}{b-a}\geq 0
$$
possiamo concludere che ogni punto $\mathrm{d}\mathrm{e}\mathrm{l}\mathrm{l}'$intervallo $[a,\ b]$ \`{e} ottenibile come \textbf{mistura} degli estremi dell'intervallo.
\noindent
Pertanto $g$ \`{e} convessa se per ogni $a, b\in J$ e per ogni $\alpha\in [0, 1]$ risulta
$$
g(\alpha a+(1-\alpha)b)\leq\alpha g(a)+(1-\alpha)g(b)
$$
Ne segue per induzione
$$
g(\alpha_{1}x_{1}+\cdots+\alpha_{n}x_{n})\leq\alpha_{1}g(x_{1})+\cdots+\alpha_{n}g(x_{n})
$$
per ogni $x_{1}$,..., $x_{n}\in J$ e $\alpha_{1}$,... , $\alpha_{n}\geq 0$ tali che $\alpha_{1}+\cdots+\alpha_{n}=1$
\end{oss}

\begin{lem} $g:J\mapsto \mathbb{R}$ convessa. Allora:
\begin{itemize}
\item Per ogni punto $a$ di $J$ esiste un numero reale $b$ tale che:
$$
g(x)\geq b(x-a)+g(a) \, \mathrm{per} \, \mathrm{ogni}\, x \in J
$$
\item $g$ \`{e} continua e quindi $B\cap J$-Borel misurabile
\end{itemize}
\end{lem}
\subsubsection{Disuguaglianza di Jensen}
\begin{thm}[dim] \textbf{Disuguaglianza di Jensen}
Dato un intervallo aperto $J$ (limitato o no) $\subset \mathbb{R}$, siano $g: J\rightarrow \mathbb{R}$ una funzione convessa e $X :\Omega\mapsto J \subset \mathbb{R}$ una $\mathrm{v}.\mathrm{a}.$ con speranza matematica. Allora $\mathrm{E}(X)\in J$. Inoltre, la $\mathrm{v}.\mathrm{a}.$ $g(X)$ ammette speranza matematica $\mathrm{e}$ risulta $g(\mathrm{E}(X))\leq \mathrm{E}(g(X))$, cioè
$$
g\left[\int_{\Omega}Xd\mathrm{P}\right]\leq\int_{\Omega}g(X)d\mathrm{P}
$$
\end{thm}


\subsection{Varianza di una variabile aletaoria}
\begin{defn}
Data una $\mathrm{v}.\mathrm{a}.$ $X$ con speranza matematica finita, la \textbf{varianza} di $X$ è il {\it momento centrale secondo}:
$$
Var (X)=\mathrm{E}((X-\mathrm{E}(X))^{2})
$$
\end{defn}

\subsection{Proprietà della varianza}
\begin{thm} [dim] Sia $X$ una $\mathrm{v}.\mathrm{a}$. con speranza matematica finita. Allora:
\begin{itemize}
\subsubsection{Def alternativa}
\item[$\star$] $Var({\it X}) =\mathrm{E}(X^{2})-\mathrm{E}(X)^{2}$; 
\subsubsection{Varianza di una trasformata affine lineare}
\item[$\star \star $] $\mathrm{V}\mathrm{a}\mathrm{r}(\alpha X+\beta)=\alpha^{2}\mathrm{V}\mathrm{a}\mathrm{r}(X)$ per ogni $\alpha, \beta$ reali; 
\subsubsection{Varianza nulla}
\item Var(X) $=0$ se e solo se $X=\mathrm{E}(X)$ (P-q.c.);
\subsubsection{Disuguaglianza di Bienaymé - $\check{\mathrm{C}}\mathrm{e}\mathrm{b}\mathrm{i}\check{\mathrm{c}}\mathrm{e}\mathrm{v}$}
\item[$\star \star \star$] Se $\epsilon>0$, allora \\ $\displaystyle \mathrm{P}\mathrm{r}(|X-\mathrm{E}(X)|\geq\epsilon)\leq\frac{\mathrm{V}\mathrm{a}\mathrm{r}(X)}{\epsilon^{2}}$ disuguaglianza di Bienaym$\acute{\mathrm{e}}-\check{\mathrm{C}}\mathrm{e}\mathrm{b}\mathrm{i}\check{\mathrm{c}}\mathrm{e}\mathrm{v}$
\end{itemize}
\end{thm}

\subsection{Covarianza di una variabile aletaoria}

\begin{defn}
Considerate due $\mathrm{v}.\mathrm{a}.\, X, Y$ con speranza matematica finita $\mathrm{e}$ tali che esista finita anche la speranza matematica del loro prodotto $\mathrm{E}(XY)$, la covarianza di $X$ e $Y$ è la differenza:
$$
\mathrm{C}\mathrm{o}\mathrm{v}(X, Y)=\mathrm{E}(XY)-\mathrm{E}(X)\mathrm{E}(Y)
$$
\end{defn}

\subsection{Proprietà della covarianza}
\begin{thm}[dim] Riesce che:
\begin{itemize} 
\subsubsection{Commutativa}
\item  $\mathrm{C}\mathrm{o}\mathrm{v}(X, Y)=\mathrm{C}\mathrm{o}\mathrm{v}(Y,X)$, e in particolare, $\mathrm{V} \mathrm{a} \mathrm{r}(X) =\mathrm{C}\mathrm{o}\mathrm{v}(X, X)$
\subsubsection{Def alternativa}
\item $\mathrm{C}\mathrm{o}\mathrm{v}(X,Y)=\mathrm{E}((X-\mathrm{E}(X))(Y-\mathrm{E}(Y)))$
\end{itemize}
\end{thm}

\begin{thm}[dim solo se c'è $\star$] Siano $X, Y, Z$ $\mathrm{v}.\mathrm{a}$. con speranza matematica finita e con $\mathrm{E}(XY)$ e  $\mathrm{E}(YZ)$ finite. Allora:
\begin{itemize}
\item[$\star$] $\mathrm{C}\mathrm{o}\mathrm{v}(\alpha X,\ Y)=\alpha \mathrm{C}\mathrm{o}\mathrm{v}(X,\ Y)$ per ogni numero reale $\alpha$;
\item[$\star$] $\mathrm{C}\mathrm{o}\mathrm{v}(X+Z,
Y)=\mathrm{C}\mathrm{o}\mathrm{v}(X,\ Y)+\mathrm{C}\mathrm{o}\mathrm{v}(Y,\ Z)$ ;
\end{itemize}

\subsubsection{Disuguaglianza di Cauchy-Schwarz}
\begin{itemize}
\item[$\star$] $\mathrm{C}\mathrm{o}\mathrm{v}(X, Y)^{2}\leq \mathrm{V}\mathrm{a}\mathrm{r}(X)\mathrm{V}\mathrm{a}\mathrm{r}(Y)$ (Disuguaglianza di Cauchy-Schwarz).\\

\noindent
Inoltre, se $X_{1}, \ldots, X_{n}$ sono $\mathrm{v}.\mathrm{a}$. con speranza matematica finita $\mathrm{e}$ tali che $\mathrm{E}(X_{i}X_{j})$ finita per ogni $i\neq j$, si ha
$$
\mathrm{Var} (\displaystyle \sum_{i=1}^{n}X_{i})=\sum_{i=1}^{n} \mathrm{Var} (X_{i})+2\displaystyle \sum_{i=1}^{n-1}\sum_{j>i}\mathrm{C}\mathrm{o}\mathrm{v}(X_{i},\ X_{j}) .
$$
\end{itemize}
\end{thm}

\noindent
In tutto le  disuguaglianze da sapere sono:
\begin{itemize}
    \item Markov
    \item Jensen
    \item Cebicev
    \item Chauchy-Schwartz
\end{itemize}

\subsection{Indice di correlazione di Bravais}
\begin{defn} Chiamiamo \textbf{indice di correlazione di Bravais} il seguente rapporto
$$
\displaystyle \rho_{X,Y} ={\displaystyle \frac {\mathrm{Cov} (X,Y)}{\sigma _{X}\sigma _{Y}}}
$$
\end{defn}

\noindent
Oss. Da Chauchy-Schwartz, purché entrambe le varianze siano >0 e finite segue che $\displaystyle \rho_{X,Y}\leq 1$ da cui:
\begin{itemize}
    \item $\displaystyle \rho_{X,Y}=-1$ allora X (o Y) è trasformata affine decrescente dell'altra (P-q.c.)
    \item $\displaystyle \rho_{X,Y}=1$ allora X (o Y) è trasformata affine crescente dell'altra (P-q.c.)
    \item $\displaystyle \rho_{X,Y}=0$ allora X e Y sono \textit{non correlate}
    \item $-1<\displaystyle \rho_{X,Y}<0$  allora sono correlate negativamente
    \item $0<\displaystyle \rho_{X,Y}<1$ allora sono correlate positivamente
    \item se vale l'uguale allora è \textit{quasi certo} che una v.a. è trasformata affine dell'altra con coeff $>0$ se Cov$>$0 (idem per $<0$)
\end{itemize}

\subsection{Comonotonia}
\begin{defn}Due $\mathrm{v}.\mathrm{a}$. $X, Y$ si dicono \textbf{comonotone} se
$$
[X(\omega)-X(\omega')][Y(\omega)-Y(\omega')]\geq 0
$$
per ogni $\omega, \omega'.$
\end{defn}

\noindent
Sono comonotone se hanno lo circa stesso andamento rispetto alla crescenza/decrescenza  dell'altra: ad esempio se una v.a. cresce l'altra non decresce.

\subsubsection{Teor su v.a. comonotone}
\begin{thm}[dim]
Siano $X,Y$ v.a. comonotone con $E(X),E(Y), E(XY)$ finite. Allora $\mathrm{Cov}(X,Y)\geq0$
\end{thm}

\noindent
La comonotonia assicura che non possono essere correlate negativamente 

\newpage
%==================================
%DISPENSA 3
%===================================
\section{PROBABILITA' 2}
\subsection{Misura prodotto}
\subsubsection{Misura sigma-finita}
\begin{defn}
Una misura $\mathrm{m}$ è $\sigma$\textbf{-finita} se esiste una partizione discreta $(A_{n})_{n\geq 1}$ di $\Omega$ con $\mathrm{m}(A_{n})<+\infty$ per ogni $n.$
\end{defn}

\noindent
 Misure $\sigma$-finite sono le misure finite (in particolare la probabilità), la misura di Lebesgue unidimensionale e  le misure di conteggio indotte da insiemi discreti.\\
 
 
\noindent
 Dato $\Omega_{i}\neq\emptyset$, siano $\mathcal{A}_{i}$ una $\sigma$ algebra su $\Omega_{i}, \mathrm{m}_{i}$ una misura $\sigma$-finita su $\mathcal{A}_{i}$ $(i=1,2)$ e posto $\Omega=\Omega_{1}\times\Omega_{2}$, consideriamo la:
 
 \subsection{Famiglia dei rettangoli misurabili}
 \begin{defn} famiglia dei \textbf{rettangoli misurabili}
 $$
 \mathcal{R}=\{\mathrm{A}_{1}\times \mathrm{A}_{2}\ :\ \mathrm{A}_{i}\in \mathcal{A}_{i}(i=1,2)\}
 $$
 \end{defn}
 
 \subsection{Sigma-algebra prodotto}
 \begin{defn} $\sigma$\textbf{-algebra prodotto}: 
 $$
 \mathcal{A}_{1}\otimes \mathcal{A}_{2}=\sigma(\mathcal{R}) (\mathrm{d}\mathrm{i}\ \mathcal{A}_{1}\, \mathrm{e} \, \mathcal{A}_{2})
 $$
\end{defn}
 
\noindent
Considerate infine, per ogni $S \subset \Omega$ la:

\begin{defn}
\textbf{sezione di $S$ relativa a} $\omega_{1}\in\Omega_{1}$:
$$
S(\omega_{1})=\{\omega_{2}\in\Omega_{2}:(\omega_{1}, \omega_{2})\in S\}
$$
\end{defn}
 
 \begin{defn}
\textbf{sezione di $S$ relativa a} $\omega_{2}\in\Omega_{2}$:
$$
S(\omega_{2})=\{\omega_{1}\in\Omega_{1}:(\omega_{1}, \omega_{2})\in S\},
$$
\end{defn}
 
 
 Ora verrà introdotta una misura sulla $\sigma$-algebra $\mathcal{A}=\mathcal{A}_{1}\otimes \mathcal{A}_{2}$ tale che verifica le seguenti due proprietà (suggerite dalla nozione di area delle figure piane $\mathrm{e}$ del relativo Principio di Cavalieri):\\
 
\noindent
(a) la misura di un rettangolo misurabile $\mathrm{A}_{1}\times \mathrm{A}_{2}$ di “$\mathrm{b}\mathrm{a}\mathrm{s}\mathrm{e}$ $\mathrm{A}_{1}$ e $ “\mathrm{a}\mathrm{l}\mathrm{t}\mathrm{e}\mathrm{z}\mathrm{z}\mathrm{a}$'' $\mathrm{A}_{2}$ è uguale al prodotto $\mathrm{m}_{1}(\mathrm{A}_{1})\mathrm{m}_{2}(\mathrm{A}_{2})$ ;\\

\noindent
(b) due insiemi $A_{1}, A_{2}\in \mathcal{A}$ hanno misura uguale se, per qualche $i\in\{1,2\},$ danno luogo a sezioni di misura uguale in corrispondenza ad ogni scelta dell'elemento che le individua in $\Omega_{i}$, cioè se $\mathrm{m}_{j}(A_{1}(\omega_{i}))=\mathrm{m}_{j}(A_{2}(\omega_{i}))$ $(j\neq i)$ per ogni $\omega_{i}\in\Omega_{i}.$

\subsection{Unicità della misura prodotto}
 \begin{lem}[dim x casa]  Sia $A\in \mathcal{A}_{1}\otimes \mathcal{A}_{2}$. Allora:
 \begin{itemize}
 \item $A(\omega_{i})\in \mathcal{A}_{j}$ per ogni $\omega_{i}(i=1,2)$ ;
 \item $\omega_{i}\mapsto \mathrm{m}_{j}(A(\omega_{i}))$ è $\mathcal{A}_{i}$-Borel misurabile $(i=1,2;j\neq i)$ 
 \end{itemize}
 \end{lem}
 
 \begin{defn}Definiamo sulla $\sigma$-algebra prodotto la misura prodotto di $\mathrm{m}_{1}$ e $\mathrm{m}_{2}$: 
 $$
 \mathrm{m}_{1}\times \mathrm{m}_{2} \ (A)=\int_{\Omega_{1}}\mathrm{m}_{2}(A(\omega_{1}))\mathrm{m}_{1}(d\omega_{1})
 $$
 \end{defn}

\begin{thm}[dim] Si dimostra che $\mathrm{m}_{1}\times \mathrm{m}_{2}(A)$ è una misura.
\end{thm}
 
\noindent 
 Il prossimo teorema assicura che la misura prodotto è l'unica misura sulla $\sigma$-algebra prodotto che verifica le proprietà (a) e (b) dell'altra pagina.

 \begin{thm}Risulta:
 \begin{itemize}
 \item $\mathrm{m}_{1}\times \mathrm{m}_{2}(\mathrm{A}_{1}\times \mathrm{A}_{2})=\mathrm{m}_{1}(\mathrm{A}_{1})\mathrm{m}_{2}(\mathrm{A}_{2})$ per ogni $\mathrm{A}_{1}\times \mathrm{A}_{2}$;
 \item $\displaystyle \mathrm{m}_{1}\times \mathrm{m}_{2}(A)=\int_{\Omega_{2}}\mathrm{m}_{1}(A(\omega_{2}))\mathrm{m}_{2}(d\omega_{2})$ per ogni $A\in \mathcal{A}_{1}\otimes \mathcal{A}_{2}.$
 \end{itemize}
 \end{thm}
 
 \begin{thm} Risulta inoltre che qualunque sia la misura $\mathrm{m}'$ sulla $\sigma$-algebra prodotto tale che $\mathrm{m}'|_{\mathcal{R}}= \mathrm{m}_{1}\times \mathrm{m}_{2}|_{\mathcal{R}}$, riesce $\mathrm{m}'=\mathrm{m}_{1}\times \mathrm{m}_{2}.$
 \end{thm}
 
 I seguenti teoremi sono due “pietre miliari'' della teoria dell'integrazione in quanto consentono di \textbf{ridurre} l' \textbf{integrale doppio} $$
\int_{\Omega}fd\mathrm{m}_{1}\times \mathrm{m}_{2}=\int_{\Omega}f(\omega_{1},\ \omega_{2})\mathrm{m}_{1}\times \mathrm{m}_{2}(d\omega_{1}\times d\omega_{2})
$$
ad un \textbf{integrale iterato} come pure di invertire l'ordine delle integrazioni successive.

\subsection{Teor di Tonelli}
Il seguente teorema di Tonelli dice che  l'integrale di una funzione non negativa sul prodotto di due spazi sigma finiti rispetto alla misura prodotto, coincide con l'integrale iterato rispetto alle due misure.

\begin{thm} \textbf{Teorema di Tonelli}
Sia $f\geq 0$, $\mathcal{A}_{1}\otimes \mathcal{A}_{2}$-Borel misurabile. Allora, per ogni $\mathrm{A}_{1}$ e $\mathrm{A}_{2}$, le funzioni:
$$
\omega_{1}\rightarrow\int_{\mathrm{A}_{2}}f(\omega_{1}, \cdot \  )d\mathrm{m}_{2}$$ 

\noindent
$$\omega_{2}\rightarrow\int_{\mathrm{A}_{1}}f( \, \cdot, \omega_{2})d\mathrm{m}_{1}
$$ 
sono, rispettivamente, $\mathcal{A}_{1}$-Borel misurabile e $\mathcal{A}_{2}$-Borel misurabile.\\
Risulta inoltre che:

\begin{align*}
\int_{\mathrm{A}_{1}\times \mathrm{A}_{2}}f \, d(\mathrm{m}_{1}\times \mathrm{m}_{2})
& =\int_{\mathrm{A}_{1}}\left (\int_{\mathrm{A}_{2}}f(\omega_{1},\ \cdot ) \ d\mathrm{m}_{2}\right)\mathrm{m}_{1} (d\omega_{1})\\
& =\int_{\mathrm{A}_{2}}\ \left(\int_{\mathrm{A}_{1}}f\ (\cdot ,\omega_{2}) \ d\mathrm{m}_{1}\right)\mathrm{m}_{2}(d\omega_{2})\ 
\end{align*}
\end{thm}

\subsection{Teor di Fubini}
Il seguente teorema di Fubini dice che se l'integrale iterato ha valore finito, allora  il valore dell'integrale è indipendente dall'ordine di integrazione.

\noindent
Esso dice che se l'integrale esiste a meno di una costante (q.o.) allora vale la formula di riduzione.

\begin{thm} \textbf{Teorema di Fubini}
Sia $f\mathrm{m}_{1}\times \mathrm{m}_{2}$-sommabile. Risulta:

$$\mathrm{A}_{1}^{(f)}\subseteq\{ \omega_{1} : f(\omega_{1},\ \cdot) \ \mathrm{è} \ \mathrm{m}_{2}-\mathrm{sommabile}\} \in \mathcal{A}_{1}$$

$$\mathrm{A}_{2}^{(f)}\subseteq \{ \omega_{2} : f(\cdot \ , \omega_{2}) \ \mathrm{è} \ \mathrm{m}_{1}-\mathrm{sommabile}\} \in \mathcal{A}_{2}$$

e $\mathrm{m}_{i}\left((\mathrm{A}_{i}^{(f)})^{c}\right)=0$ $(i=1,2)$.

\noindent
Inoltre, per ogni $\mathrm{A}_{i}(i=1,2)$, le funzioni:
$$
g_{1}:\omega_{1}\mapsto\left\{\begin{array}{ll}
\displaystyle \int_{\mathrm{A}_{2}}f(\omega_{1},\ \cdot)d\mathrm{m}_{2} & \mathrm{s}\mathrm{e}\ \omega_{1}\in \mathrm{A}_{1}^{(f)}\\
0 & \mathrm{s}\mathrm{e}\ \omega_{1}\not\in \mathrm{A}_{1}^{(f)}
\end{array}\right.
$$

$$
g_{2}:\omega_{2}\mapsto\left\{\begin{array}{ll}
\displaystyle \int_{\mathrm{A}_{1}}f(\cdot \ , \omega_{2})d\mathrm{m}_{1} & \mathrm{s}\mathrm{e}\ \omega_{2}\in \mathrm{A}_{2}^{(f)}\\
0 & \mathrm{s}\mathrm{e}\ \omega_{2}\not\in \mathrm{A}_{2}^{(f)}
\end{array}\right.
$$
\end{thm}

\noindent
 La conclusione dei due teoremi è la stessa ma le ipotesi sono molto diverse.



\subsection{Estensione dei risultati a 3 o più variabili}
La nozione di misura prodotto ed $\mathrm{i}$ relativi teoremi di Tonelli $\mathrm{e}$ Fubini possono essere estesi al caso di più di due fattori.

\begin{defn} Siano
\begin{itemize}
\item $\mathcal{A}_{h}$ una $\sigma$-algebra su un insieme $\Omega_{h}\neq \emptyset$ 
\item $\mathrm{m}_{h}$ una misura $\sigma$-finita su $\mathcal{A}_{h}(h=1,\ \ldots,\ m)$
\item $\Omega=\Omega_{1}\times\cdots\times\Omega_{m}$
\end{itemize}
\end{defn}

\subsubsection{Rettangoli misurabili}
\begin{defn} Definiamo la famiglia dei \textbf{rettangoli misurabili}:
$$
\mathcal{R}^{(m)}=\{\mathrm{A}_{1}\times\cdots\times \mathrm{A}_{m}\ :\ \mathrm{A}_{h}\in \mathcal{A}_{h} \, \, (h=1,\ \ldots,\ m)\}
$$
\end{defn}

\subsubsection{Sigma-algebra prodotto m-dimensionale}
\begin{defn} Definiamo la $\sigma$-algebra prodotto $\mathcal{A}_{1}\otimes\cdots\otimes \mathcal{A}_{m}=\sigma(\mathcal{R}^{(m)})$

\end{defn}

\subsubsection{Misura prodotto}
\begin{thm}
Esiste una sola misura $\mathrm{m}=\mathrm{m}_{1}\times\cdots\times \mathrm{m}_{m}$ sulla $\sigma$-algebra prodotto, detta \textbf{misura prodotto di} $( \mathrm{m}_{1},\ \ldots,\ \mathrm{m}_{m})$, tale $\mathrm{c}\mathrm{h}\mathrm{e}$:
$$
\displaystyle \mathrm{m}_{1}\times\cdots\times \mathrm{m}_{m}(\mathrm{A}_{1}\times\cdots\times \mathrm{A}_{m})=\prod_{h=1}^{m}\mathrm{m}_{h}(\mathrm{A}_{h})$$
$per \; ogni \; rettangolo \; misurabile \;  \mathrm{A}_{1}\times\cdots\times \mathrm{A}_{m}$ 

\end{thm}

\subsubsection{Teor di Fubini e Tonelli m-dimensionali}
\begin{thm} Sussiste la generalizzazione dei teoremi di Tonelli e Fubini che permette di ridurre l'\textbf{integrale multiplo} ad un \textbf{integrale iterato}:
$$
\displaystyle \int_{\Omega}f \ d(\mathrm{m}_{1}\times \cdots \displaystyle \times \mathrm{m}_{m})=\int_{\Omega}f(\omega_{1},\ \ldots,\ \omega_{m})\mathrm{m}_{1}\times \cdots \times \mathrm{m}_{m}(d\omega_{1}\times\ \cdots\ \times d\omega_{m}).
$$
\end{thm}

\noindent
Il seguente teorema permette di invertire l'ordine delle integrazioni successive
\begin{thm}
Sia $f$ $\mathrm{m}$-sommabile. Data una permutazione $h_{1}, \ldots, h_{m}$ di $\{$1, $\ldots, m\}$ si ha
$$
\displaystyle \int_{\Omega}fd\mathrm{m}=\int_{\Omega_{h_{1}}}\left(\cdots\left(\int_{\Omega_{h_{m}}}f(\omega_{1},\ \ldots,\ \omega_{m})\mathrm{m}_{h_{m}}(d\omega_{h_{m}})\right)\cdots\right)\mathrm{m}_{h_{1}}(d\omega_{h_{1}}),
$$ 
dove, nel caso che non sia $f\geq 0$, ogni integrale
$$
\displaystyle \int_{\Omega_{h_{k}}}\left(\cdots\left(\int_{\Omega_{h_{m}}}f(\omega_{1},\ \ldots,\ \omega_{m})\mathrm{m}_{\mathrm{h}_{\mathrm{m}}}(d\omega_{h_{m}})\right)\cdots \right)\mathrm{m}_{h_{k}}(d\omega_{h_{k}}) \,\,\,\, (k>1)
$$
esiste per ogni $(\omega_{h_{1}},\ \ldots,\ \omega_{h_{k-1}})$ non appartenente ad un insieme $\mathrm{m}_{h_{k-1}}$ -trascurabile $\mathrm{e}$ viene esteso (per effettuare l'integrazione successiva relativa alla misura $\mathrm{m}_{h_{k-1}}$) su tale insieme dandogli valore zero.
\end{thm}

\subsubsection{Sigma-algebra di Borel m-dimensionale}
\begin{defn} Definiamo la \textbf{$\sigma$-algebra di Borel} di $(\mathbb{R}^{*})^{m}$ come la $\sigma$-algebra prodotto $\mathcal{B}^{(m)}=\overbrace{\mathcal{B} \otimes  \cdots \otimes \mathcal{B}}^{\textrm{m volte}}$ che: \\
\noindent
[ - ] come nel caso unidimensionale $(\mathbb{R}^{*})^{m}$ può essere vista come la $\sigma$-algebra generata dalla famiglia degli insiemi aperti di m-uple, oppure dagli insiemi $S_{x_1,...,x_m} = ]-\infty,x_1]\times ... \times ]-\infty,x_m]$ ($x_1,...,x_m \in \mathbb{R}$)\\
\noindent
[ - ] è costituita dai \textbf{boreliani} di $(\mathbb{R}^{*})^{m}$
\end{defn}

\subsubsection{Misura di Lebesgue m-dimensionale}
\begin{defn} Definiamo la \textbf{misura di Lebesgue} m\textbf{-dimensionale} come la misura prodotto 
$$
\lambda^{(m)}=\overbrace{\lambda \times ... \times \lambda}^{\textrm{m volte}}
$$
\end{defn}



\subsection{Coppia aleatoria}
\begin{defn} Denotiamo :

    con $\aleph_{i}$ la $\sigma$-algebra su $\mathfrak{X}_{i}\neq\emptyset$ $(i=1,2)$

  con $X_{i}:\Omega\mapsto \mathrm{X}_{i}(i=1,2)$

  con $\mathfrak{X}=\mathfrak{X}_{1}\times \mathfrak{X}_{2}$ 
 
 con $\aleph=\aleph_{1}\otimes\aleph_{2}$
\end{defn}

\begin{defn}
 $\textbf{X}=(\mathrm{X}_{1}, \mathrm{X}_{2}):\omega\mapsto(X_{1}(\omega), X_{2}(\omega))$   \textbf{coppia aleatoria}. Cioé ogni ente aleatorio su $\Omega$ a valori in $\mathfrak{X}$
\end{defn}


\begin{thm}[dim x casa] \textbf{X} \`{e} una coppia aleatoria se e solo se $X_{i}$ \`{e} un ente aleatorio a valori in $\mathfrak{X}_{i}$  $(i=1,2)$ .
\end{thm}

\subsection{Legge congiunta}
\begin{defn}
\textbf{Legge} (o \textbf{distribuzione}) \textbf{congiunta} di $X_{1}, X_{2}$ \`{e} la legge $\mathrm{P}_{\mathrm{X}}$ di \textbf{X}
\end{defn}

\subsection{Densità congiunta}
\begin{defn}
Sia $\mu_{i}$ misura $\sigma$-finita di riferimento su $\aleph_{i}(i=1,2)$, allora chiamo
$(\mu_{1}, \mu_{2})$-\textbf{densità congiunta}  di $X_{1}, X_{2}$ ogni funzione $\aleph$-Borel misurabile $f:\mathfrak{X}\mapsto[0,\ +\infty]$ tale che \\
$\displaystyle \mathrm{P}_{\mathrm{X}}(\mathrm{A})=\mathrm{P}(\mathrm{X}\in \mathrm{A})=\int_{\mathrm{A}}fd(\mu_{1}\times\mu_{2})=\int_{\mathrm{A}}f(x_{1}, x_{2})\mu_{1}\times\mu_{2}(dx_{1}\times dx_{2})$
\end{defn}

\subsection{Legge marginale e densità marginale di $X_{i}$}
\begin{defn}
\textbf{Legge marginale} di $X_{i}$ \`{e} la legge $\mathrm{P}_{X_{i}}$; mentre la \textbf{densità marginale} di $X_{i}$  \`{e} una sua $\mu_{i}$-densit\`{a} $(i=1,2)$
\end{defn}

\noindent
Breve riepilogo su cosa è stato definito:
\noindent
$\Omega \;\; \overset{X_1}{\longrightarrow} \;\; \mathfrak{X}_1\\
\mathcal{A} \;\;\;\;\;\;\; \,\,\,\;\;\; \aleph_1  \,\,\,\;\;\;$\\
$P \;\;\;\;\;\;\;\;\;\;\; P_{X_1}$ / $\mu_1=\left\{\begin{matrix}
f\geq 0 \\ 
P_{X_1}(A \in \aleph)=\int_A f d\mu 
\end{matrix}\right.$ \\
\\
$\Omega \;\; \overset{X_2}{\longrightarrow} \;\; \mathfrak{X}_2\\
\mathcal{A} \;\;\;\;\;\;\; \,\,\,\;\;\; \aleph_2  \,\,\,\;\;\;$\\
$P \;\;\;\;\;\;\;\;\;\;\; P_{X_2}$ / $\mu_2=\left\{\begin{matrix}
f\geq 0 \\ 
P_{X_2}(A \in \aleph)=\int_A f d\mu 
\end{matrix}\right.$
\\


\subsection{Funzione di ripartizione congiunta}
\begin{defn} Definiamo la \textbf{Funzione di ripartizione congiunta}
delle v.a. $ X_{1}, X_{2}$, la funzione:
$$
F_{\mathrm{X}}(x_{1},\ x_{2})=\mathrm{P}_{\mathrm{X}}\left(]-\infty,\ x_{1}]\times]-\infty,\ x_{2}]\right)=\mathrm{P}((X_{1},\ X_{2})\leq(x_{1},\ x_{2}))
$$
\end{defn}

\subsection{Teor fondamentale del calcolo delle probabilità (bivariato)}
\begin{thm}\textbf{Teorema fondamentale del calcolo delle probabilità (caso bivariato} Sia $g:\mathrm{X}\mapsto \mathbb{R}^{*}$ una funzione sommabile rispetto alla legge con- giunta di $X_{1} \, \mathrm{e} \, X_{2}$. Allora, posto $\mu=\mu_{1}\times\mu_{2}$, si ha

$$\displaystyle \int_{\{\mathrm{X}\in \mathrm{A}\}}g(X_{1}, X_{2})d\mathrm{P}=\left\{\begin{array}{l}
\displaystyle \int_{\mathrm{A}}gd\mathrm{P}_{\mathrm{X}}\\
\\
\displaystyle \int_{\mathrm{A}}gfd\mu
\end{array}\right.$$
per ogni $\mathrm{A}\in\aleph$
\end{thm}

\begin{thm} [dim] Risulta:
\begin{itemize}
\item[$\bullet $]  $\mathrm{P}_{X_{1}}(\mathrm{A}_{1})=\mathrm{P}_{\textbf{X}}(\mathrm{A}_{1}\times \mathfrak{X}_{2})$, $\mathrm{P}_{X_{2}}(\mathrm{A}_{2})=\mathrm{P}_{\textbf{X}}(\mathfrak{X}_{1}\times \mathrm{A}_{2})$ per ogni $\mathrm{A}_{i}\in\aleph_{i}$; 
\item[$\bullet \bullet$] 
Se $f$ \`{e} una $(\mu_{1}, \mu_{2})$-densit\`{a} congiunta di $X_{1}, X_{2}$, la funzione:\\
$ f_{X_{i}}(x_{i})=\left\{\begin{array}{ll}
\displaystyle \int_{\mathfrak{X}_{2}}f(x_{1}, x_{2}) \, \mu_{2}(dx_1) & \mathrm{s}\mathrm{e}\ i=1\\
\\
\displaystyle \int_{\mathfrak{X}_{1}}f(x_{1}, \, x_{2})\mu_{1}(dx_2) & \mathrm{s}\mathrm{e}\ i=2
\end{array}\right.$\\ 
per ogni $x_{i}\in \mathrm{X}_{i},$ \`{e} una $\mu_{i}$-densit\`{a} marginale di $X_{i}(i=1,2)$ .
\end{itemize}
\end{thm}


\subsection{Indipendenza di enti aleatori}
Due enti aleatori $X_{1}$ e $X_{2}$ sono \textbf{indipendenti} se
$$
\mathrm{P}_{\mathrm{X}}(\mathrm{A}_{1}\times \mathrm{A}_{2})=\mathrm{P}_{X_{1}}(\mathrm{A}_{1})\mathrm{P}_{X_{2}}(\mathrm{A}_{2})
$$
per ogni rettangolo misurabile $\mathrm{A}_{1}\times \mathrm{A}_{2}.$

\subsubsection{Teor su  equivalenza  indipendenza fra e.a.}
\begin{thm}
Sono equivalenti le proposizioni:
\begin{itemize}
\item[]$\bullet X_{1}$ e $X_{2}$ sono indipendenti;
\item[] $\bullet\bullet$ $\mathrm{P}_{\mathrm{X}}=\mathrm{P}_{X_{1}}\times \mathrm{P}_{X_{2}}.$
\item[] $\bullet \bullet \bullet$ Se $f_{\mathrm{X}}$ \`{e} una $(\mu_{1},\ \mu_{2})$-densit\`{a} congiunta di $X_{1}$ e $X_{2}$, la proposizione $\bullet$ \`{e} equivalente alla: $f_{\mathrm{X}}(x_{1},\ x_{2})=f_{X_{1}}(x_{1})f_{X_{2}}(x_{2})$, a meno di un insieme $\mu_{1}\times\mu_{2}$-trascurabile
\end{itemize}
\end{thm}

\subsubsection{Teor su  indipendenza di e.a. trasformati}
\begin{thm}[dim]
Siano $\mathcal{G}_{i}$ $\sigma$-algebra su $\mathfrak{Y}_{i}$ e $g_{i}: \mathfrak{X}_{i}\mapsto \mathfrak{Y}_{i}$ un'applicazione $(\aleph_{i}, \mathcal{G}_{i})-$ misurabile $(i=1,2)$. Siano inoltre $X_{1}, X_{2}$ indipendenti. Allora, sono pure indipendenti gli enti aleatori trasformati $g_{1}(X_{1})$ e $g_{2}(X_{2})$.
\end{thm}

\subsubsection{Teor su conseguenza  indipendenza v.a. su E()}
\begin{thm} [dim] Siano $X_{1}, X_{2}$ v.a. indipendenti. Allora, se $X_{i}\geq 0$ $ (i=1,2)$ o  $\mathrm{E}(\mathrm{X})$ finita $(i=1,2)$, esiste la speranza matematica del prodotto $\mathrm{e}$ si ha
$$
\mathrm{E}(X_{1}X_{2})=\mathrm{E}(X_{1})\mathrm{E}(\mathrm{X}_2)
$$
\end{thm} 

\noindent
Oss. Estensioni ad m-uple\\
Le definizioni date si estendono ``pari pari'' alle $m$-ple aleatorie $(m\geq 3)$ e rimangono validi tutti $\mathrm{i}$ risultati ottenuti.

\subsubsection{Teor su sottoinsieme di e.a. indipendenti}
\begin{thm}
Se gli enti aleatori $X_{1}, \ldots, X_{m}$ sono \textbf{indipendenti}, allora lo sono pure quelli ottenuti operando una selezione tra di essi. 
\end{thm}

\subsubsection{Indipendenza successione di eventi}
\begin{defn}
I termini della successione $X_{1}, X_{2}, \ldots$ di enti aleatori si dicono \textbf{indipendenti} se sono indipendenti $X_{k_{1}}, \ldots, X_{k_{n}}$ (ocio) per ogni scelta di $k_{1}, \ldots, k_{n}$ distinti $\mathrm{e}$ di $n.$
\end{defn}

\newpage
%==================================
%DISPENSA 4
%===================================

\section {VARIABILI ALEATORIE}

I  vari tipi di convergenza di seguito, che vedremo, sono via via più deboli, (la convergenza puntuale dell'analisi matematica non rientra nell'elenco perché è più forte di tutte)
\begin{itemize}
    \item [1] (convergenza certa)
    \item [2] convergenza quasi certa: ha unicità q.c. del limite
    \item [3] convergenza in probabilità: ha unicità q.c. del limite 
    \item [4] convergenza in distribuzione
\end{itemize}
\subsection{Convergenza quasi certa}
\begin{defn}  
Una successione $(X_{n})_{n\geq 1}$ di $\mathrm{v}.\mathrm{a}$. converge \textbf{quasi certamente} alla v.a. $X$ se: $X_{n}\rightarrow X$ ;
$$\mathrm{ (P-q.c.) },  \mathrm{cioè } \;
\mathrm{P}(X_{n} \rightarrow X)=1,$$

\noindent
$\mathrm{o}\mathrm{v}\mathrm{e}$  $\{X_{n}\rightarrow X\}=\{\omega\ :\ X_{n}(\omega)\rightarrow X(\omega)\}$ è l'\textbf{insieme di convergenza}
\end{defn}

\subsubsection{Teor caratterizzazione cqc}
\begin{thm}[dim] Le seguenti proposizioni sono equivalenti:
\begin{itemize}
\item [i)]$X_{n}\rightarrow X$ $ (\mathrm{P}- \mathrm{q}.\mathrm{c}.)$ ;
\item [ii)] $\displaystyle \lim_{n\rightarrow+\infty}\mathrm{P}(\bigcap_{m\geq n}\{|X_{m}-X|<\epsilon\})=1$ qualunque sia $\epsilon>0$; 
\item [iii)] $\displaystyle \lim_{n\rightarrow+\infty}\mathrm{P}(\bigcap_{m\geq n}\{|X_{m}-X|\leq\epsilon\})=1$ qualunque sia $\epsilon>0.$
\end{itemize}
\end{thm}


\subsubsection{Teor unicità q.c. del limite e convergenza in probabilità della trasformata continua}
\begin{thm}[dim]
Sia $X_{n}\rightarrow X (\mathrm{P}- \mathrm{q}.\mathrm{c}.)$ . Risulta:
\begin{itemize}
    \item Se $X_{n}\rightarrow Y$ (P-q.c.), allora $X=Y$ (P-q.c.);
    \item Se $g:\mathbb{R}\rightarrow \mathbb{R}$ è una funzione continua, allora $g(X_{n})\rightarrow g(X)$ (P-q.c.).
\end{itemize}
\end{thm}

\subsection{Convergenza in probabilità}
\begin{defn}
Una successione $(X_{n})_{n\geq 1}$ di $\mathrm{v}.\mathrm{a}$. converge in probabilità alla $\mathrm{v}.\mathrm{a}.$ $X$ (in simboli $X_{n}\rightarrow^{\mathrm{p}\mathrm{r}}X$) se per ogni $\epsilon>0$ risulta\\ $$\displaystyle \lim_{n\rightarrow+\infty}\mathrm{P}(|X_{n}-X|<\epsilon)=1$$
oppure
$$
\lim_{n\rightarrow+\infty}\mathrm{P}(|X_{n}-X|\leq\epsilon)=1
$$
\end{defn}

\subsubsection{Teor cqc implica cip}
\begin{thm}[dim]
La convergenza quasi certa implica quella in probabilità.
\end{thm}

\subsubsection{Teor di caratterizzazione cip}
\begin{thm}[dim] Le seguenti proposizioni sono equivalenti: 
\begin{itemize}
\item [i)] $X_{n}\rightarrow^{\mathrm{p}\mathrm{r}}X$;
\item [ii)] Ogni sottosuccessione di $(X_{n})_{n\geq 1}$ ammette una sottosuccessione convergente $\mathrm{P}$-quasi certamente a $X.$
\end{itemize}
\end{thm}

\subsubsection{Teor unicità q.c. del limite e convergenza in probabilità della trasformata continua}
\begin{thm}[dim] Sia $X_{n}\overset{Pr}{\rightarrow}X$. Risulta:
\begin{itemize}
    \item [(i)]  Se $X_{n}\overset{Pr}{\rightarrow}Y$, allora $X=Y$ (P-q.c.);
    \item [(ii)] Se $g:\mathbb{R}\rightarrow \mathbb{R}$ è una funzione continua, allora $g(X_{n})\overset{Pr}{\rightarrow}g(X)$ .
\end{itemize}
\end{thm}

\subsection{Leggi dei grandi numeri}
Le leggi dei grandi numeri forniscono una formulazione matematica precisa all'intuizione empirica che:
\begin{itemize}
\item  la media aritmetica di un gran numero di realizzazioni di $\mathrm{v}.\mathrm{a}$. indipendenti ed equidistribuite è verosimilmente vicina alla comune speranza matematica;
\item la frequenza relativa di successo (rapporto tra il numero di eventi che si verificano tra quelli considerati $\mathrm{e}$ la numerosità di questi ultimi) di un gran numero di prove indipendenti ed equiprobabili è verosimilmente vicina alla comune probabilità.
\end{itemize}

\subsubsection{Somma di v.a.}
\begin{defn}
Date le $\mathrm{v}.\mathrm{a}. X_{1}, \ldots, X_{n}$, poniamo
$$
S_{n}=X_{1}+\cdots+X_{n}
$$
Se $\mathrm{E}(X_{1}), \ldots, \mathrm{E}(X_{n})$ sono finite, allora
$$
\mathrm{E}\left(\frac{S_{n}}{n}\right)=\frac{\mathrm{E}(X_{1})+\cdots+\mathrm{E}(X_{n})}{n}
$$
\end{defn}

\subsubsection{Somma di v.a. indicatrici}
\begin{defn}
 Dati gli eventi $A_{1}, \ldots, A_{n}$, poniamo
$$
S_{n}=I_{A_{1}}+\cdots+I_{A_{n}}
$$
Allora
$$
\mathrm{E}\left(\frac{S_{n}}{n}\right)=\frac{\mathrm{P}(A_{1})+\cdots+\mathrm{P}(A_{n})}{n}
$$
\end{defn}

\subsection{Legge debole dei grandi numeri}
\begin{defn}
Una successione $(X_{n})_{n\geq 1}$ di $\mathrm{v}.\mathrm{a}.$ verifica la\textbf{ legge debole dei grandi numeri}
se
$$
\frac{S_{n}}{n}-\mathrm{E}\left (\frac{S_{n}}{n}\right)\rightarrow 0
$$
\end{defn}

\subsubsection{Teor di Markov (cond. suff.)}
\begin{thm} \textbf{Teorema di Markov}\; (dim).
 Sia la successione di $\mathrm{v}.\mathrm{a}. $ $(X_{n})_{n\geq 1}$ tale che:
 $$
 \displaystyle \lim_{n\rightarrow+\infty}\frac{\mathrm{V}\mathrm{a}\mathrm{r}(S_{n})}{n^{2}}=0 \,\,\,\, (condizione \,\,\, di \,\,\,Markov).$$
\noindent
Allora, la successione verifica la legge debole dei grandi numeri.
\end{thm}

\subsubsection{Teor di Cebicev (altra condiz suff)}
\begin{thm}[dim]
Sia la successione di $\mathrm{v}.\mathrm{a}. (X_{n})_{n\geq 1}$ tale che $\mathrm{C}\mathrm{o}\mathrm{v}(X_{i}, X_{j})\leq 0 $ $(i\neq j)$ e
\begin{center}
$\displaystyle \lim_{n\rightarrow+\infty}\frac{\mathrm{V}\mathrm{a}\mathrm{r}(X_{1})+\cdots+\mathrm{V}\mathrm{a}\mathrm{r}(X_{n})}{n^{2}}=0$. $\,\,\,\,\,\,\,\,\, (3)$
\end{center}
Allora $(X_{n})_{n\geq 1}$ verifica la legge debole dei grandi numeri.
\end{thm}

\subsubsection{Teor su condizione suff per altra condiz suff}
\begin{oss}
L'ipotesi (3) sussiste se è verificata una delle due condizioni:

$\displaystyle \bullet\lim_{n\rightarrow+\infty}\frac{\mathrm{V}\mathrm{a}\mathrm{r}(X_{n})}{n}=0$;

$\displaystyle \bullet\sum_{n\geq 1}\frac{\mathrm{V}\mathrm{a}\mathrm{r}(X_{n})}{n^{2}}<+\infty$
\end{oss}

\noindent
Dunque sussiste la legge debole dei grandi numeri, per successioni di $\mathrm{v}.\mathrm{a}.$ non positivamente correlate, se le varianze $\sigma_{n}^{2}=\mathrm{V}\mathrm{a}\mathrm{r}(X_{n}) (n\geq 1)$ sono equilimitate oppure non divergono troppo velocemente.

\subsection{Legge forte dei grandi numeri}
\begin{defn}
Una successione $(X_{x})_{n\geq 1}$ di v.a. verifica la \textbf{legge forte dei grandi numeri} se
$$\displaystyle \frac{S_{n}}{n}-\mathrm{E}\left(\frac{S_{n}}{n}\right)\rightarrow 0 \ (P-q.c.)$$ 
\end{defn}

\subsubsection{Teorema di Rajchman}
\begin{thm}\textbf{Teorema di Rajchman}.
 Sia $\mathrm{C}\mathrm{o}\mathrm{v}(X_{i}, X_{j})\leq 0 \, (i\neq j)\, e \,$ le varianze $\mathrm{V}\mathrm{a}\mathrm{r}(\mathrm{X}_n) (n\geq 1)$ equilimitate. Allora, $(X_{n})_{n\geq 1}$ verifica la legge forte dei grandi numeri
\end{thm}

\subsubsection{Corollario teor di Rajchman}
\begin{cor} Risulta:
\begin{itemize}
\item [i)] [dim] Siano $X_{1}, X_{2}, \ldots$ indipendenti, equidistribuite $\mathrm{e}$ con varianza finita. Allora $\, \displaystyle \frac{S_{n}}{n}\rightarrow \mathrm{E}(\mathrm{X})$ (P-q.c.)

\item [ii)] (Borel) Siano gli eventi $A_{1}, A_{2}, \ldots$ indipendenti ed equiprobabili. Allora $\, \displaystyle \frac{S_{n}}{n}\rightarrow \mathrm{P}(\mathrm{A})$ (P-q.c.) 
\end{itemize}
\end{cor}

\subsection{Convergenza in distribuzione}

La convergenza puntuale di una successione di $\mathrm{v}.\mathrm{a}$. ad una $\mathrm{v}.\mathrm{a}$. non assicura la convergenza delle rispettive funzioni di ripartizione alla funzione di ripartizione della $\mathrm{v}.\mathrm{a}$. limite.

\begin{defn}
Una successione $(X_{n})_{n\geq 1}$ di $\mathrm{v}.\mathrm{a}$. \textbf{converge in distribuzione} alla $\mathrm{v}.\mathrm{a}.\,  X$ (in simboli $X_{n} \overset{di}{\rightarrow} X$) se $F_{n}(x)\rightarrow F(x)$ in ogni punto di continuità $x$ di $F$ avendo indicato, rispettivamente, con $F$ ed $F_{n}$ le funzioni di ripartizione delle $\mathrm{v}.\mathrm{a}$. $X$ e  $X_{n}$ $(n\geq 1)$ .
\end{defn}

Da un punto di vista intuitivo, la differenza sostanziale tra la convergenza quasi certa, quella in probabilità $\mathrm{e}$ la convergenza in distribuzione può essere espressa dicendo che nelle prime due $X_{n}$ \textbf{tende ad essere uguale} a $X$ con alta probabilità, mentre nella convergenza in distribuzione $X_{n}$ \textbf{tende ad avere la medesima legge} di $X.$

\subsubsection{Teor link fra conv. in prob e conv. in distrib}
\begin{thm}[dim]
La convergenza in probabilità implica quella in distribuzione.
\end{thm}

\subsubsection{Teor di caratterizzazione cid (Portmanteau) }
\begin{thm}[\textbf{Portmanteau Theorem}] 
Le seguenti proposizioni sono equivalenti:
\begin{itemize}
    \item $X_{n}\rightarrow^{\mathrm{d}\mathrm{i}} X$;
    \item $\displaystyle \lim_{n\rightarrow+\infty}\mathrm{E}(f(X_{n}))=\mathrm{E}(f(X))$, per ogni funzione $f$ : $\mathbb{R}\rightarrow \mathbb{R}$ continua $\mathrm{e}$ limitata.
\end{itemize}
\end{thm}


\subsubsection{Teor unicità in distribuzione e convergenza in distribuzione della trasformata continua}
 (Ocio) Unicità non q.c., ma  in legge
\begin{thm}[dim]
Sia $X_{n}\overset{di}{\rightarrow} X$. Sussistono allora le seguenti proposizioni:

(i) Se $X_{n}\overset{di}{\rightarrow} Y$, allora $X, Y$ sono equidistribuiti (cioè $\mathrm{P}_{X}=\mathrm{P}_{Y}$);

(ii) Se $g:\mathbb{R}\rightarrow \mathbb{R}$ è una funzione continua, allora $g(X_{n})\overset{di}{\rightarrow} g(X)$ .
\end{thm}

\noindent
 Osservazione. La convergenza in distribuzione non assicura la convergenza del momento centrale $m$-simo delle v.a. della successione al momento centrale $m$-simo della v.a. limite. Ciò avviene solo se la v.a. è limitata

\subsubsection{Teor di Polya}
\begin{thm}\textbf{Teorema di Polya}  Se la successione di funz. rip. $(F_{n})_{n\geq 1}$, converge ad una funz. rip. $\mathrm{F}$ continua, allora la convergenza è uniforme.
\end{thm}

\subsubsection{Somma ridotta}
Date le v.a. $X_{1},\ldots, X_{n}$ con speranza matematica $\alpha$ e varianza $\sigma^{2}>0$ finita, si chiama somma ridotta di $X_{1}, \ldots, X_{n}$ la seguente trasformata della v.a. $S_{n}$:
$$
S_{n}^{*}=\frac{S_{n}-\alpha n}{\sqrt{\sigma^{2}n}}.
$$
Chiaramente, $\mathrm{E}(S_{n}^{*})=0$. Inoltre, se $X_{1}, \ldots, X_{n}$ sono indipendenti, si ha Var $(S_{n}^{*})=1.$

\subsection{Teor limite centrale}
\begin{thm}[\textbf{Teorema limite centrale} ]
Siano le v.a. $X_{1}, X_{2}, \ldots$ indipendenti ed equidistribuite con varianza finita non nulla. Allora, la successione $(F_{S_{n}^{*}})_{n\geq 1}$ delle funz. rip. delle somme ridotte converge uniformemente alla funz. rip. della distribuzione normale $\mathrm{N}(0,1)$.
\end{thm}

\noindent
In altri termini, la successione $(S_{n}^{*})_{n\geq 1}$ converge in distribuzione a una qualsiasi v.a. distribuita secondo la distribuzione normale $\mathrm{N}(0,1)$.

\subsubsection{Teor disuguaglianza di Berry-Essén}
Nelle ipotesi del teorema limite centrale assumendo la finitezza e positività del momento centrale assoluto terzo $\beta=E(|X_1-E(X_1)|^3)$, risulta che il massimo errore di approssimazione fra la funzione di ripartizione delle somme ridotte rispetto alla funzione di ripartizione della normale è dato da
$\displaystyle \sup_{x\in \mathbb{R}}|F_{S_{n}^{*}}(x)-\Phi(x)|\leq\frac{K\beta}{\sqrt{(\sigma^{2})^{3}}}\frac{1}{\sqrt{n}}$$ (\textbf{disuguaglianza di Berry-Essén})\\
\noindent
\mathrm{con} $\displaystyle \frac{\sqrt{10}+3}{6\sqrt{2\pi}}\approx 0.409732\leq K\leq 0.4785$ (allo stato attuale della conoscenza)\\

\subsubsection{Approssimazione normale}
\noindent
\textbf{Approssimazione normale} Il teorema limite centrale giustifica l' approssimazione, che viene usualmente fatta nelle applicazioni pratiche, della legge di $S_{n}$ con quella della distribuzione normale $\mathrm{N}(0,1)$ nel caso di indipendenza ed equidistribuzione. Approssimazione $\mathrm{c}\mathrm{h}\mathrm{e}$, indicata con $\Phi$ la funzione di ripartizione della normale $\mathrm{N}(0,1)$, si basa sulla seguente relazione
$$
\mathrm{P}(S_{n}\leq x)=\mathrm{P}(S_{n}^{*}\leq\frac{x-\alpha n}{\sqrt{\sigma^{2}n}})\approx\Phi(\frac{x-\alpha n}{\sqrt{\sigma^{2}n}})=\frac{1}{\sqrt{2\pi}}\int_{-\infty}^{\frac{x-\alpha n}{\sqrt{\sigma^{2}n}}}\mathrm{e}^{-\frac{t^{2}}{2}}dt.
$$
dove $E(X_n)=\alpha$ e $Var(X_n)=\sigma^2 <+\infty$ $\forall n$\\

\noindent
L'esperienza empirica suggerisce che la soglia di applicabilità di questa ``approssimazione normale''sia per $n$ tra 30 e 50 per la maggior parte delle distribuzioni impiegate nella pratica che non siano troppo asimmetriche (valori $\mathrm{c}\mathrm{h}\mathrm{e},$ è bene tenere presente, non sono nè suggeriti, nè giustificati da alcun risultato teorico). Nel caso di distribuzioni molto asimmetriche $\mathrm{i}$ valori 30, 50 devono essere aumentati. Ad esempio, se consideriamo una successione di eventi indipendenti di medesima probabilità $p$, una buona approssimazione la si ottiene prendendo $p(1-p)n\geq 5$. Osserviamo inoltre $\mathrm{c}\mathrm{h}\mathrm{e}$, nel caso di v.a. a valori interi, si ottiene, in generale, una migliore approssimazione di $\mathrm{P}(S_{n}\leq x)$ sostituendo in $\displaystyle \frac{x-\alpha n}{\sqrt{\sigma^{2}n}}$ il valore $x$ con $x+\displaystyle \frac{1}{2}.$\\

\noindent
Sempre in questo ambito, una ulteriore approssimazione normale di impiego usuale nella pratica è la seguente:
$$
\mathrm{P}\left(\left|\frac{S_{n}}{n}-\alpha\right|\geq\epsilon\right)\approx 2\Phi\left(-\epsilon\sqrt{\frac{n}{\sigma^{2}}}\right)=\sqrt{\frac{2}{\pi}}\int_{\epsilon\sqrt{\frac{n}{\sigma^{2}}}}^{+\infty}\mathrm{e}^{-\frac{t^{2}}{2}}dt
$$
che fornisce una approssimazione, per $n$ sufficientemente grande, della proba- bilità che la media aritmetica delle prime $n$ v.a. si discosti dalla comune speranza matematica $\alpha$ per almeno di $\epsilon>0$

\subsection{Teorema di Glivenko-Cantelli}

Le leggi dei grandi numeri $\mathrm{e}$ il teorema limite centrale considerano il comportamento asintotico, rispettivamente, della media aritmetica $\mathrm{e}$ della somma ridotta. \\ 
\noindent
Il prossimo teorema fornisce invece una formulazione matematica precisa all'intuizione empirica che la funzione di ripartizione che distribuisce la probabilit\`{a} in modo uguale su un gran numero di realizzazioni di v.a. indipendenti ed equidistribuite \`{e} verosimilmente prossima alla comune funzione di ripartizione.

\noindent
Date le v.a. $X_{1}$, \ldots , $X_{n}$, la funzione di dominio $\mathbb{R}\times\Omega$, con $\omega$ fissato e $x$ che varia:
\begin{align*}
F^{(X_{1},\ldots,X_{n})}(x,\ \omega) &=\ 
\frac{I_{\{X_{1}\leq x\}}(\omega)+\cdots+I_{\{X_{n}\leq x\}}(\omega)}{n}\\
& = \frac{{I_{]-\infty,x]}(X_{1}(\omega))+\cdots+I_{]-\infty,x]}(X_{n}(\omega))}}{n}, 
\end{align*}
\`{e} detta \textbf{funzione di ripartizione empirica} (relativa a $ X_1, \ldots , X_{n}$).\\
\noindent
-$ F^{(X_{1},\ldots,X_{n})} (\cdot, \omega)$ \`{e} una funzione di ripartizione per ogni $\omega\in\Omega$ e quindi, dipendendo dai casi elementari, \`{e} una funzione di ripartizione aleatoria; risulta infatti:\\

\noindent
-$F^{(X_{1},\ldots,X_{n})}(x, \omega)\leq F^{(X_{1},\ldots,X_{n})}(x', \omega)$, se $x<x'$ (poich\`{e} $I_{]-\infty,x]}\leq I_{]-\infty,x']}$);\\

\noindent
-$F^{(X_{1},\ldots,X_{n})}(x, \omega)\leq F^{(X_{1},\ldots,X_{n})}(x', \omega)$, se $x<x'$ (poich\`{e} $I_{]-\infty,x]}\leq I_{]-\infty,x']}$);\\

\noindent
-$F^{(X_{1},\ldots,X_{n})}(x, \omega)=0$, se $x<\displaystyle \min(X_{1}(\omega),\ \ldots,\ X_{n}(\omega))$;\\

\noindent
 -$F^{(X_{1},\ldots,X_{n})}(x, \omega)=1$, se $x\displaystyle \geq\max(X_{1}(\omega),\ \ldots,\ X_{n}(\omega))$ ;\\
 
 \noindent
-$F^{(X_{1},\ldots,X_{n})}(x^{+}, \omega)=F^{(X_{1},\ldots,X_{n})}(x,\ \omega)$ (in quanto $\displaystyle \lim_{t\rightarrow x^{+}}I_{\{X_{i}\leq t\}}=I_{\{X_{i}\leq x\}}$).\\



\begin{thm}Teorema di Glivenko-Cantelli\\
 Siano le v.a. $X_{1}, X_{2}, \ldots$ indipendenti con comune funzione di ripartizione $F$. Allora
$$\displaystyle \sup_{x\in \mathbb{R}} \left| F^{(X_{1},\ldots,X_{n})}(x,\ \cdot)-F(x) \right|\rightarrow 0 \,\,\,\, (P-q, \mathrm{c}.),
$$
\mathrm{cioé}
$\mathrm{P(}\{\omega\in\Omega$ : $F^{X_{1},\ldots,X_{n})} (\cdot, \omega$) converge uniformemente a F\}$\mathrm{)} =1$
\end{thm}

\newpage
%==================================
%DISPENSA 5
%===================================
\section{MEDIE CONDIZIONATE}

\subsection{Condizionamento a eventi non trascurabili}

\begin{defn}
Supposto $P(H)>0$ poniamo
$$
\mathrm{P}(A|H)=\frac{P(A \cap H)}{P(H)}\,\, \,\,\,\, \forall A \in  \mathcal{A}
$$
\end{defn}

\begin{thm}
Si dimostra che essa è ancora una probabilità.
\end{thm}

\noindent
Allora $\mathrm{P}(\cdot|H)$ è una probablità, detta \textbf{probabilità condizionata} a $H$. 






\begin{thm}[dim]
Sia $X$ una $\mathrm{v}.\mathrm{a}.$. Allora, $X$ è $\mathrm{P}(\cdot|H)$-sommabile se $\mathrm{e}$ solo se $XI_{H}$ è $\mathrm{P}$-sommabile. Inoltre, nel caso di sommabilità,
$$ \displaystyle
\int_{\Omega}Xd\mathrm{P}(\cdot|H)=\frac{\displaystyle \int_{H}XdP}{\mathrm{P}(H)}
$$
\end{thm}

\begin{defn}
Data una $\mathrm{v}.\mathrm{a}.$ $X$ con speranza matematica, la\textbf{ speranza matematica condizionata di $X$ a $H$} viene identificata con l'elemento della retta reale ampliata:
$$
\mathrm{E}(X|H)=\int_{\Omega}Xd\mathrm{P}(\cdot|H)
$$
Allora,
$$
\displaystyle \mathrm{E}(X|H)=\frac{\displaystyle \int_{H}XdP}{\mathrm{P}(H)}=\frac{\mathrm{E}(XI_{H})}{\mathrm{P}(H)}
$$
\end{defn}

\subsection{Condizionamento a sigma-algebre}
\begin{defn} Definisco alcune quantità utili nel seguito:
\begin{itemize}
    \item   \textbf{Informazione}   è una v.a. $X$ non osservabile
    \item   \textbf{Informazione ottenibile} $\mathcal{H}$ è la $\sigma$-algebra generata dagli \textbf{eventi osservabili} (es. $\sigma(\mathcal{H})$)
    \item una v.a. \textbf{osservabile} è costruibile attraverso indicatori ed elementi di $\mathcal{H}$ (es. $Y=\sum_{i=1}^4 y_i I_{H_i}$)
\end{itemize}
\end{defn}
%\begin{defn}
%$\bullet$ Informazione ottenibile: la $\sigma$-algebra: $
%\mathcal{H}=\{\bigcup_{j\in J}H_{j})
%$ degli eventi osservabili
%\end{defn}
\noindent
$ Y$ è $\mathcal{H}$-Borel misurabile 

\noindent
Dunque la $\mathrm{v}.\mathrm{a}.$ $Y$ consente di calcolare la speranza matematica condizionata di $X$ a un qualsiasi evento osservabile non trascurabile. \\


\noindent
Appare quindi ragionevole ritenere che l'assunzione congiunta della $\mathcal{H}$-Borel misurabilità $\mathrm{e}$ della $\displaystyle \int_H XdP=\int_H YdP$ possa essere intesa come caratterizzante la speranza matematica condizionata all'informazione ottenibile (rappresentata, nel caso particolare in esame, dalla $\sigma$-algebra generata $\mathcal{H}$).\\

\noindent
\textbf{Formulazione generale}\\
Adottando questo punto di vista $\mathrm{e}$ passando a una formulazione generale, consideriamo, per descrivere formalmente {\it l}'{\it informazione ottenibile} tramite un prefissato {\it processo} $di$ {\it osservazione}, una $\sigma$-algebra $\mathcal{H}\subseteq \mathcal{A}, \mathrm{i}$ cui elementi chiameremo \textbf{eventi osservabili} (in quanto rappresentano gli eventi aleatori il cui valore di verità è acquisibile tramite il processo di osservazione).\\

\noindent
L'assunzione che la famiglia degli eventi osservabili $\mathcal{H}$ sia una $\sigma$-algebra appare del tutto naturale una volta osservato che l'evento certo è osservabile, che la negazione di un evento osservabile è osservabile $\mathrm{e}$ che la disgiunzione di eventi osservabili è osservabile.

\subsubsection{Versione della speranza matematica condizionata}
\begin{defn}
Data una  $\mathrm{v}.\mathrm{a}.$  $X$ con $\mathrm{E}(X)$ finito, chiamiamo \textbf{versione (della speranza matematica condizionata di $X$ a $\mathcal{H}$)} (in simboli $\mathrm{E}(X|\mathcal{H})$) ogni $\mathrm{v}.\mathrm{a}$. estesa $Y$ tale che:
\begin{itemize}
\item [(a)] sia $\mathcal{H}$-Borel misurabile;
\item [(b)] per ogni $H\in \mathcal{H}$, risulti
$$
\int_{H}Xd\mathrm{P}=\int_{H}Yd\mathrm{P}.
$$
\end{itemize} 
\end{defn}

\begin{itemize}
    \item[$\bullet$] $Y$ è una $\mathrm{v}.\mathrm{a}.$ osservabile.  Pertanto, una volta risolta l'incertezza sugli eventi osservabili viene individuato il ``vero'' valore di $Y.$  
\end{itemize}


\begin{thm}[dim lampo] $Y$ è finita (P-q.c.) in quanto $\mathrm{E}(Y)=\mathrm{E}(X)$.
\end{thm}
\begin{thm}[dim] Sia $Y'$ un'altra versione. Allora $Y=Y'$ (P-q.c.).
\end{thm}

\noindent
Dunque la speranza matematica condizionata è definita a meno di eventi osservabili di probabilità nulla.\\

\noindent Perciò con un \textbf{abuso di notazione}, useremo il simbolo $\mathrm{E}(X|\mathcal{H})$ per denotare anche una sua qualsiasi versione.


\begin{defn} Un evento osservabile $H$ si chiama \textbf{$\mathrm{P}$-atomo} (in breve \textbf{atomo}) se è un evento di probabilità positiva non ripartibile in due eventi osservabili ancora di probabilità positiva $(\mathrm{i}\mathrm{n}$ altri termini, $\mathrm{p}\mathrm{e}\mathrm{r}$ ogni $H_{1}\subset H\ \mathrm{s}\mathrm{i}\ \mathrm{h}\mathrm{a}\ \mathrm{P}(H_{1})=0$ o $ \mathrm{P}(H_{1})=\mathrm{P}(H))$ .
\end{defn}

\noindent
Sia $(H_{i})_{i\in I}$ una partizione discreta di $\Omega$ costituita da eventi non trascurabili. Allora gli atomi della $\sigma$-algebra generata coincidono con gli elementi della partizione.

\begin{thm}
Sia $H$ un $\mathrm{P}$-atomo di $\mathcal{H}$. Esiste allora $H_{1}\subseteq H$ tale che $\mathrm{P}(H_{1})= \mathrm{P}(H)\mathrm{ \;e \; }\mathrm{E}(X|\mathcal{H})(\omega)=\mathrm{E}(X|H)$ per ogni $\omega\in H_{1}.$
\end{thm}

\noindent
L'ipotesi di atomicità non è in generale rimovibile.
\subsubsection{Proprietà della speranza matematica condizionata}
\begin{thm} Risulta:
\begin{itemize}
    \item $ \mathrm{E}(\mathrm{E}(X|\mathcal{H}))=\mathrm{E}(X)$
    \item $ \mathrm{E}(X|\{\emptyset,\ \Omega\})=\mathrm{E}(X)$. Qualora l'informazione ottenibile sia quella: minima (costituita solamente dagli eventi certo $\mathrm{e}$ impossibile) (\textbf{informazione nulla}), di $X$ possiamo conoscere unicamente la speranza matematica;
    \item $\mathrm{E}(X|\mathcal{A})=X$ (P-q.c.). Qualora l'informazione ottenibile sia quella: massima (consistente nel sapere quale caso elementare è vero), di $X$ possiamo conoscere il suo ``vero'' valore. 
    \item $ \mathrm{E}(X|\mathcal{H})=X$ (P-q.c.), se $X$ è $\mathcal{H}$-Borel misurabile
\end{itemize}
\end{thm}


\begin{thm} Siano $X, Y, X_{1}, X_{2}$, etc. $\mathrm{v}.\mathrm{a}$. con speranza matematica finita. Valgono le proposizioni seguenti nelle quali le relazioni riguardanti speranze matematiche condizionate devono intendersi sussistere $\mathrm{P}$-quasi certamente.
\begin{itemize}
    \item $\textsc{monotonia}: \mathrm{E}(X|\mathcal{H})\leq \mathrm{E}(Y|\mathcal{H})$, se $X\leq Y$ $(\mathrm{P}-\mathrm{q}.\mathrm{c}.);$
    
    \item $\mathrm{E}(X|\mathcal{H})=\mathrm{E}(Y|\mathcal{H})$, se $X=Y$ (P-q.c.);
    
    \item $\mathrm{E}(X|\mathcal{H})=\mathrm{E}(Y|\mathcal{H})$, se $X=Y$ (P-q.c.);
    
    \item \textsc{internalità}: $a\leq \mathrm{E}(X|\mathcal{H})\leq b$, se $a\leq X\leq b$ (P-q.c.);
    
    \item  \textsc{internalità stretta}: $a<\mathrm{E}(X|\mathcal{H})<b$, se $a<X<b$ (P-q.c.);
    \item $\textsc{linearità}: \displaystyle \mathrm{E}(\sum_{i=1}^{n}\alpha_{i}X_{i}|\mathcal{H})=\sum_{i=1}^{n}\alpha_{i}\mathrm{E}(X_{i}|\mathcal{H})$ qualunque siano $\mathrm{i}$ nu- meri reali $\alpha_{1}, \ldots, \alpha_{n}$;
    
    \item  $|\mathrm{E}(X|\mathcal{H})|\leq \mathrm{E}(|X||\mathcal{H})$ ;
    \item $ \textsc{convergenza monotona}: \mathrm{E}(X_{n}|\mathcal{H})\uparrow \mathrm{E}(X|\mathcal{H})$, se $X_{n}\uparrow X\mathrm{\; e \;}X_{1}\geq0$ $(\mathrm{P}- \mathrm{q}.\mathrm{c}.)\ ;$
    \item  $\displaystyle \mathrm{E}(\sum_{n\geq 1}\alpha_{n}X_{n}|\mathcal{H})=\sum_{n\geq 1}\alpha_{n}\mathrm{E}(X_{n}|\mathcal{H})$ , se $(\alpha_{n})_{n\geq 1}$ è una successione di numeri reali non negativi, $X_{n}\geq 0$ per ogni $n\mathrm{\; e}$ la speranza matematica della serie $\displaystyle \sum_{n\geq 1}\alpha_{n}X_{n}$ è finita;
    \item $ \textsc{convergenza dominata}: \mathrm{E}(X_{n}|\mathcal{H})\rightarrow \mathrm{E}(X|\mathcal{H})$ , se $|X_{n}|\leq Z$ (P-q.c.) per ogni $n, \mathrm{E}(Z)$ finita $\mathrm{e \;}X_{n}\rightarrow X$;
    
    \item $\mathrm{E}(ZX|\mathcal{H})=Z\mathrm{E}(X|\mathcal{H})$ , se $Z$ è $\mathcal{H}$-Borel misurabile $\mathrm{e}$ limitata; assicura che le $\mathrm{v}.\mathrm{a}.$ osservabili $\mathrm{e}$ limitate possono essere ``estratte'' dall'operatore di speranza matematica condizionata.
    
    \item Se $\mathcal{K}$ è una sotto $\sigma$-algebra di $\mathcal{H}$, allora $\mathrm{E}(\mathrm{E}(X|\mathcal{H})|\mathcal{K})=\mathrm{E}(X|\mathcal{K})$ e $\mathrm{E}(\mathrm{E}(X|\mathcal{K})|\mathcal{H})=\mathrm{E}(X|\mathcal{K})\ .$
    \end{itemize}
\end{thm}

\subsubsection{Teorema di Lévy}
\begin{thm}[$\star \star $ \textbf{Teorema di Lévy}]
Sia $(\mathcal{H}_{n})_{n\geq 1}$ una successione non decrescente di $\sigma$-algebre tali che
$$
\mathcal{A}=\sigma(\bigcup_{n\geq 1}\mathcal{H}_{n})
$$
(\textbf{struttura informativa}). Allora, per ogni $\mathrm{v}.\mathrm{a}.$ $X$ con speranza matematica finita risulta
$$
\mathrm{E}(X|\mathcal{H}_{n})\rightarrow X\ (\mathrm{P}- \mathrm{q}.\mathrm{c}.)
$$
\end{thm}

\begin{thm}[dim]
Siano $X$ una $\mathrm{v}.\mathrm{a}$. con varianza finita  e $Y$ una versione limitata di $\mathrm{E}(X|\mathcal{H})$. Allora
$$
\mathrm{V}\mathrm{a}\mathrm{r} (X)=\mathrm{V}\mathrm{a}\mathrm{r}(Y)+\mathrm{E}((X-Y)^{2})\geq \mathrm{V}\mathrm{a}\mathrm{r}(Y)
$$
\end{thm}

\subsection{Funzione di regressione}
In moltissime situazioni l'informazione ottenibile consiste nell'osservare il va- lore di un ente aleatorio $X$ : $(\Omega,\ \mathcal{A})\rightarrow(\mathfrak{X},\ \aleph),$ ove $ \aleph$ contiene $\mathrm{i}$ singoletti.

\noindent
Poichè l'informazione deriva dall'osservazione di $X,$ viene naturale identi- ficare la $\sigma$-algebra degli eventi osservabili $\mathcal{H}$ con la $\sigma$-algebra indotta
$$
X^{-1}(\aleph)=\{X^{-1}(\mathrm{A}):\ \mathrm{A}\ \in\aleph\}\subseteq \mathcal{A}
$$
da $X$ su $\Omega$ (da un punto di vista interpretativo, la conoscenza dei valori di verità di tutti gli eventi di tale $\sigma$-algebra equivale alla conoscenza del vero valore di $X$ in quanto $\{X=y\}=\{X\in\{y\}\}\in X^{-1}(\aleph)$ per ogni $y\in \mathfrak{X}$).


\begin{lem}[dim solo caso $Z$ semplice] Sia $Z$ una $\mathrm{v}.\mathrm{a}.$ $X^{-1}(\aleph)$-Borel misurabile. Esiste allora una funzione $\aleph$-Borel misurabile $g:\mathrm{X}\mapsto \mathbb{R}^{*}$ tale che $Z=g(X)$ .
\end{lem}

\noindent
Data una $\mathrm{v}.\mathrm{a}. \; Y$ con speranza matematica finita,
$$
\mathrm{E}(Y|X)=\mathrm{E}(Y|X^{-1}(\aleph))
$$
è la speranza matematica condizionata di $Y\mathrm{a}\mathrm{l}\mathrm{l}'$osservazione di $X.$

\subsubsection{Indipendenza fra v.a. Y ed e.a. X}
\begin{thm}[$\star \star$ dim]
Se la $\mathrm{v}.\mathrm{a}.$ $ Y$ $ \mathrm{e}$ l'ente aleatorio $X$ sono \underline{indipendenti}, allora
$$
\mathrm{E}(Y|X)=\mathrm{E}(Y)=\mathrm{E}(Y|\{\emptyset,\ \Omega\})
$$
cioè l'informazione proveniente da $X$ equivale all'informazione nulla.
\end{thm}


\begin{defn}
Chiamiamo \textbf{funzione di regressione di $Y$ su $X$} ogni funzione
$$
\mathrm{E}(Y|X=\cdot):\mathfrak{X}\mapsto \mathbb{R}^{*}
$$
tale $\mathrm{c}\mathrm{h}\mathrm{e}$:
\begin{itemize}
\item [1.] sia $\mathrm{P}_{X}$-integrabile;
\item [2.] per ogni $\mathrm{A}\in\aleph$ risulti:
$$
\int_{\{X\in \mathrm{A}\}}Yd\mathrm{P}=\int_{\mathrm{A}}\mathrm{E}(Y|X=x)\mathrm{P}_{X}(dx)
$$
\end{itemize}
\end{defn}


$\bullet$ \`{e} finita $(\mathrm{P}_{X}-\mathrm{q}.\mathrm{c}.)$\\

$\bullet$ \`{e} definita a meno di insiemi $\mathrm{P}_{X}$-trascurabili\\

$\bullet (dim) \; \mathrm{E}(Y|X=X(\cdot))$ \`{e} una versione di $\mathrm{E}(Y|X)$ .

\begin{thm}
Sia $x_{0}\in \mathrm{X}$ tale che $\mathrm{P}(X=x_{0})>0$. Allora,
$$ 
\mathrm{E}(Y|X=x_{0})=\frac{\displaystyle \int_{\{X=x_{0}\}}Yd\mathrm{P}}{\mathrm{P}(X=x_{0})}=\mathrm{E}(Y|\{X=x_{0}\}).
$$
Si noti la la profonda \textbf{differenza} concettuale delle due nozioni di \textbf{funzione di regressione} $\mathrm{e}$ di \textbf{speranza matematica condizionata all'informazione disponibile}: funzione dipendente dai {\it valori osservabili} la prima; dai {\it casi elementari} la seconda.
\end{thm}

\subsubsection{Proprietà della funzione di regressione}
\begin{thm}Siano $Y, Y_{1}, Y_{2}$ etc. v.a. con speranza matematica finita. Riesce allora:

\begin{itemize}

\item $\displaystyle \mathrm{E}(Y)=\int_{\mathfrak{X}}\mathrm{E}(Y|X=x)\mathrm{P}_{X}(dx)$ .\\
\noindent
Valgono inoltre le proposizioni seguenti nelle quali le relazioni riguardanti funzioni di regressione devono intendersi sussistere $\mathrm{P}_{X}$-quasi certamente.

\item Se $X=x_{0}\in \mathfrak{X}$ (P-q.c.), allora $\mathrm{E}(Y|X=\cdot)=\mathrm{E}(Y)$. Inoltre, se $X$ \`{e} l'applicazione identica, allora $\mathrm{E}(Y|X=\cdot)=Y$;

\item $ \mathrm{E}(\alpha I_{\Omega}|X=\cdot)=\alpha$;

\item \textsc{monotonia}: $\mathrm{E}(Y_{1}|X=\cdot)\leq \mathrm{E}(Y_{2}|X=\cdot)$ , se $Y_{1}\leq Y_{2}$ (P-q.c.);

\item $\mathrm{E}(Y_{1}|X=\cdot)=\mathrm{E}(Y_{2}|X=\cdot)$ , se $Y_{1}=Y_{2}$ (P-q.c.);

\item \textsc{internalità}: $a\leq \mathrm{E}(Y|X=\cdot)\leq b$, se $a\leq Y\leq b$ (P-q.c.);

\item  \textsc{internalità stretta}: $a<\mathrm{E}(Y|X=\cdot )<b$, se $a<Y<b \; ($P- q.c.$)$;

\item \textsc{linearità}: $\displaystyle \mathrm{E}(\sum_{i=1}^{n}\alpha_{i}Y_{i}|X=\cdot)=\sum_{i=1}^{n}\alpha_{i}\mathrm{E}(Y_{i}|X=\cdot)$ qualunque siano $\mathrm{i}$ numeri reali $\alpha_{1}, \ldots, \alpha_{n}$;

\item $|\mathrm{E}(Y|X=\cdot)|\leq \mathrm{E}(|Y||X=\cdot)$;

\item \textsc{convergenza monotona}: $\mathrm{E}(Y_{n}|X=\cdot)\uparrow \mathrm{E}(Y|X=\cdot)$, se $Y_{n}\uparrow Y\mathrm{\; e \;} Y_{1}\geq 0$ (P-q.c.);

\item $ \displaystyle \mathrm{E}(\sum_{n\geq 1}\alpha_{n}Y_{n}|X=\cdot)=\sum_{n\geq 1}\alpha_{n}\mathrm{E}(Y_{n}|X=\cdot)$ , se $(\alpha_{n})_{n\geq 1}$ \`{e} una successione di numeri reali non negativi, $Y_{n}\geq 0$ per ogni $n\mathrm{\; e}$ la speranza matematica della serie $\displaystyle \sum_{n\geq 1}\alpha_{n}Y_{n}$ \`{e} finita;

\item \textsc{convergenza dominata}: $\mathrm{E}(Y_{n}|X=\cdot)\rightarrow \mathrm{E}(Y|X=\cdot)$, se $|Y_{n}|\leq Z$ (P-q.c.) per ogni $n, \mathrm{E}(Z)$ finita $\mathrm{e \;}Y_{n}\rightarrow Y$;

\item Sia $g:\mathrm{X}\mapsto \mathbb{R}^{*}\mathrm{u}\mathrm{n}\mathrm{a}$ funzione $\aleph$-Borel misurabile tale che la v.a. $g(X)Y$ ammetta speranza matematica finita. Allora,
$$
\mathrm{E}(g(X)Y|X=\cdot)=g(\cdot) \mathrm{E}(Y|X=\cdot)\ .
$$  

In particolare, $\mathrm{E}(g(X)|X=\cdot)=g(\cdot)$ ;
\end{itemize}
\end{thm}

\noindent

\subsection{Funzione di regressione e  metodo dei minimi quadrati }
Concludiamo mostrando l'importanza fondamentale della funzione di regressione nella soluzione del problema (centrale nella problematica statistica $\mathrm{e}$ di grande interesse applicativo) di stimare, a partire dall'osservabile $X$, il non osservabile $Y$ commettendo un ``errore pi\`{u} piccolo possibile''.\\

\noindent
Consideriamo una stima $g(X)$ di $Y$. Notato $\mathrm{c}\mathrm{h}\mathrm{e}$, data una funzione continua $\mathrm{e}$ crescente $\varphi$ tale che $\displaystyle \varphi(0)=0$ $\mathrm{e}\lim_{x\rightarrow+\infty}\varphi(x)=+\infty, \mathrm{E}(\varphi(|Y-g(X)|))$ \`{e} una quantit\`{a} che tende ad essere grande, se $g(X)$ assume valori distanti da $Y, \mathrm{e}$ piccola, se $g(X)$ assume valori vicini a $Y$, viene naturale usarla per misurare la bont\`{a} dell'approssimazione di $Y$ con $g(X)$.\\

\noindent
Tra le varie scelte possibili di $\varphi$, adottiamo quella relativa al \textbf{metodo dei minimi quadrati}, cio\`{e} poniamo $\varphi(x)=x^{2}$. Giungiamo così ad intendere la frase ``errore pi\`{u} piccolo possibile'' nel senso di ``\textbf{errore quadratico medio pi\`{u} piccolo possibile}'' $\mathrm{e}$ quindi a cercare una funzione $g^{*}$ tale che
$$
\mathrm{E}((Y-g^{*}(X))^{2})\leq \mathrm{E}((Y-g(X))^{2})
$$
per ogni funzione $g$ a quadrato $\mathrm{P}_{X}$-integrabile.\\


\noindent
Il prossimo risultato collega la funzione di regressione con il metodo dei mini- mi quadrati assicurando che ogni funzione di regressione di $Y$ su $X$ fornisce un esempio di stima dei minimi quadrati di $Y.$

\begin{thm} Sia $Y$ una v.a. a quadrato integrabile. Allora, $\mathrm{E}(Y|X=X(\cdot))$ \`{e} a quadrato integrabile e risulta

\begin{align*}
\mathrm{E}((Y-\mathrm{E}(Y|X))^{2})\ &=\ \mathrm{E}((Y-\mathrm{E}(Y|X=X(\cdot))^{2})\\
&=\ \min\{\mathrm{E}((Y-g(X))^{2}):\int_{\mathrm{X}}g^{2}d\mathrm{P}_{X}<+\infty\}
\end{align*}
\end{thm}

\subsubsection{Retta di regressione lineare}
Se, in particolare, $g(x)=ax+b$ $(a>0)\mathrm{\; e \;}X, Y$  $\mathrm{v}.\mathrm{a}$. a quadrato integrabile e Var(X) $>0$ finita, si ha

\begin{align*}
f(a, b) &= \mathrm{E}((Y-(aX+b)^{2})=\mathrm{E}(Y^{2}-2(aX+b)Y+(aX+b)^{2})\\
&= \mathrm{E}(Y^{2})-2\mathrm{E}(aXY+bY)+\mathrm{E}(a^{2}X^{2}+2abX+b^{2})\\
&= \mathrm{E}(Y^{2})-2\mathrm{E}(aXY+bY)+\mathrm{E}(a^{2}X^{2}+2abX+b^{2})\\
&=\mathrm{E}(Y^{2})-2a\mathrm{E}(XY)-2b\mathrm{E}(Y)+a^{2}\mathrm{E}(X^{2})+2ab\mathrm{E}(X)+b^{2}
\end{align*}

\noindent
da cui otteniamo che il punto di minimo $(a^{\star}, b^{\star})$ \`{e} soluzione del sistema\\

$ \left\{\begin{array}{l} \displaystyle 
\frac{\partial}{\partial a}f(a,\ b)=-2\mathrm{E}(XY)+2a\mathrm{E}(X^{2})+2b\mathrm{E}(X)=0\\
\\
\displaystyle  \frac{\partial}{\partial b}f(a,\ b)=-2\mathrm{E}(Y)+2a\mathrm{E}(X)+2b=0
\end{array}\right.$\\
\\
\noindent
e quindi
 $$ \displaystyle a^{\star}=\frac{\mathrm{C}\mathrm{o}\mathrm{v}(X,Y)}{\mathrm{V}\mathrm{a}\mathrm{r}(X)}\ \;\;\;\;\;\;\;\;\;\;\; \;\; \displaystyle b^{\star}=\mathrm{E}(Y)-\frac{\mathrm{C}\mathrm{o}\mathrm{v}(X,Y)}{\mathrm{V}\mathrm{a}\mathrm{r}(X)}\mathrm{E}(X).$$

\noindent
otteniamo cos\`{i} l'espressione della trasformata affine $\mathrm{d}\mathrm{e}\mathrm{l}\mathrm{l}'$osservabile {\it X} che meglio approssima in media quadratica il non osservabile $Y$
$$
y=\frac{\mathrm{C}\mathrm{o}\mathrm{v}(X,Y)}{\mathrm{V}\mathrm{a}\mathrm{r}(X)}(x-\mathrm{E}(X))+\mathrm{E}(Y)
$$
\noindent
detta \textbf{retta di regressione lineare} di $Y$ su $X.$


\end{document}
